%\chapter{納豆菌の菌糸直径計測および菌糸本数推定手法}
%\section{納豆菌の直径計測手法}
%\section{納豆菌の菌糸本数推定手法}

\chapter{提案システム}

提案システムでは,教室などの閉鎖空間において,
個人が所有する複数のスマートデバイスの音声出力をネットワークを介し同期させ制御することで,スピーカアレイを構築する(図\ref{fig:shikumi2}).
空間内に配置した仮想音源の位置に基づき,その音源位置を囲む最寄りの3つのスマートデバイス(ノード)を設定し振幅パニングすることで,現実世界における想定位置で音源を鳴らして定位する.
このようなスピーカアレイを構築するにあたり,実空間に分布する複数のスマートデバイスの相対位置を推定するとともに,端末間での時刻同期が必要不可欠である.

\begin{figure}[p]\centering
  \hspace{-2mm}\includegraphics[clip,width=1.1\hsize]{img/shikumi3.png}
  \caption{提案システムの概念}\label{fig:shikumi2}
\end{figure}

この章では,まず,提案システムで用いた
相対位置に基づく音像定位手法について触れる.
次に,その手法を実現するための
端末間の音声パルスの到達時間差による
時刻同期手法および相対距離計測手法,
そして相対位置推定手法を述べる.
さらに,パルス圧縮を用いた信号検出手法を示し,
最後に,スピーカアレイ全体の制御手法について解説する.



\section{DBAP法を用いた音像定位}

はじめに,複数のスマートデバイスを使ってどのように音像定位するのかについて述べる.
平面配置のスピーカアレイを用いて音像定位をする手法として,
Distance-based amplitude panning (DBAP) 法\cite{dbap}がある.

DBAP法は,
任意の数のスピーカの位置が既知であり,
端末間のスピーカの出力特性が等しいときに,
仮想音源と各スピーカとの距離から距離減衰を計算することで,
各スピーカの振幅を制御して音像を合成する手法である.

図\ref{fig:DBAP}のように,位置 $(x_s,y_s)$ にある仮想音源 $VS$ から,
位置 $(x_i,y_i)$ にある端末 $i$ への距離 $d_i$ を次のように定義する.

\begin{figure}[p]\centering
  \hspace{-2mm}\includegraphics[clip,width=1.1\hsize]{img/DBAP.png}
  \caption{DBAP法の図解}\label{fig:DBAP}
\end{figure}

$$
d_i = \sqrt{(x_i - x_s)^2 + (y_i - y_s)^2} \qquad (\mathrm{for}\ 1 \leq i \leq N)
$$

DBAP法では,仮想音源の位置に関係なく,各スピーカからの音の強さの合計を

$$
I = \sum_{i=1}^N v_i^2 = 1
$$

として正規化している.
$i$ 番目のスピーカの相対的な振幅は距離に反比例するので

$$
v_i = \frac{k}{d_i^a}
$$

と定義できる.
$k$ はすべてのスピーカと仮想音源の位置に依存した係数で,

$$
k = \frac{1}{\sqrt{\sum_{i=1}^N \frac{1}{d_i^{2a}}}}
$$

である.
係数 $a$ は振幅の距離減衰係数で

$$
a = \frac{R}{20 \log_{10}2} \\
$$

と定義する.
$R$ はロールオフ係数で,受聴者と音源の距離に基づく減衰の量である.
$R=6\ [\mathrm{dB}]$ の場合は,
自由空間における距離減衰の逆二乗則に基づき,音の強さのレベルが音源からの距離が2倍になるごとに6dBずつ減少することを意味する.
また,半自由空間では $R=3\sim5\ [\mathrm{dB}]$ 程度となる.

以上のとおり,
複数のスマートデバイスが
同期的に制御でき,
端末の位置が判明しており,
端末間のスピーカの出力特性が均一であれば,
この手法を用いて音像定位ができることが分かった.



\section{時刻同期と測距}

基準となる時刻が違う二つの時間軸を持つ端末間において同期するには,
互いに音声パルスを出せばよい(図\ref{fig:beeptobeep}).

\begin{figure}[tb]\centering
  \hspace{-2mm}\includegraphics[clip,width=1.1\hsize]{img/beeptobeep.png}
  \caption{beeptobeep}\label{fig:beeptobeep}
\end{figure}

この手法はTPSN(time-sync Protocol for sensor network)\cite{tpsn}
などで提案されている.
原理を説明する.

端末 $A$ が自身の時刻 $t_0$ に音声パルスを発生すると,
そのパルスは音速で空間に広がり,
端末 $B$ 内の時刻 $t_1$ に受信される.
さらに,端末 $B$ からも端末 $B$ 内の時刻 $t_2$ に音声パルスを発すると,
このパルスも音速で空間に広がり,
端末 $A$ 内の時刻 $t_3$ に受信される.
ここで,図の通り,
端末 $A$ 内のパルス時間間隔 $t_3-t_0$ と
端末 $B$ 内のパルス時間間隔 $t_2-t_1$ には差が生じる(図\ref{fig:clocksynchronization}).

\begin{figure}[tb]\centering
  \hspace{-2mm}\includegraphics[clip,width=1.1\hsize]{img/clock_synchronization.png}
  \caption{clocksynchronization}\label{fig:clocksynchronization}
\end{figure}

パルスの往復で伝播にかかった時間は共に等しいと仮定すると,

$$
t_0' = t_1 - \frac{(t_3 - t_0) - (t_2 - t_1)}{2} \\
$$

となり,端末 $B$ 内時刻で端末 $A$ のパルスが発せられた時刻を推定することができる.
以上が時刻同期の原理である.

音速を $c$ とすれば,副次的に端末間の距離も求まる.

$$
d_{AB} = \frac{(t_3 - t_0) - (t_2 - t_1)}{2c}
$$

端末AB間で何らかの処理を同期的に実行したい場合は
図のようにして実行すべき時間を求められる(図\ref{fig:flowchart3}).

\begin{figure}[tb]\centering
  \hspace{-2mm}\includegraphics[clip,width=1.1\hsize]{img/flowchart3.png}
  \caption{同期実行}\label{fig:flowchart3}
\end{figure}

基準となる端末Aのパルスの受信時間およびその信号の伝達時間と,
それを受信してからの経過時間 $S$ をもとに,同期的に処理を実行できる.

こうして,端末間の時刻同期と距離測定ができた.



\section{相対位置推定}


次に,複数のスマートデバイスの空間分布をどう推定するかについて述べる.
次のように定式化し誤差関数を最小化する最適化問題を考える.

$$
\varepsilon(\hat{x_1}, \dots, \hat{x_N}) = \sum_{i=1}^N \sum_{j\in M(i)} \left( \| \hat{ x_i } - \hat{ x_j } \| - d_{ij} \right)^2 \\
%\DeclareMathOperator*{\argmin}{arg\,min}
(\hat{x_1} \dots \hat{x_N}) = \mathrm{argmin} \varepsilon(\hat{x_1} \dots \hat{x_N})
$$

ここで,
$N \in \mathbb{N}$ は端末の数,
$M(i) \subset \{1,\dots,N\}$ は端末 $i$ と相対距離が計測できた端末の集合,
$d_{ij} \in \mathbb{R}$ は実際に計測された端末の距離とし,
$\hat{ x_i } \in \mathbb{R}^2$ n番目の端末の位置推定値で,初期値は乱数を置く.

最急降下法を使って反復的に解く.更新式は次のようになる.

$$\begin{aligned}
\hat{x_i} (n + 1) & = \left. \hat{x_i} (n) - a \frac{\partial \varepsilon}{\partial \hat{x_i}} \right|_{\hat{x} = \hat{x}(n)} \\
\frac{\partial \varepsilon}{\partial \hat{x_i}}
&= \sum_{j\in M(i)} \frac{\partial \left( \|\hat{ x_i } - \hat{ x_j }\| - d_{ij} \right)^2}{\partial \hat{x_i}} \notag\\
&= 2 \sum_{j\in M(i)} \left( \| \hat{x_i} - \hat{x_j} \| - d_{ij} \right) \frac{\partial \| \hat{x_i} - \hat{x_j} \|}{\partial \hat{x_i}} \notag\\
&= 2 \sum_{j\in M(i)} \left( 1 - \frac{d_{ij}}{\| \hat{x_i} - \hat{x_j} \|} \right) \left( \hat{x_i} - \hat{x_j} \right).
\end{aligned}$$

$n$ は反復回数, $a$ は更新式のステップ幅である.
図\ref{fig:relpo_s}にすべての端末間で距離が取得できたとしたシミュレーション結果を示す.


\begin{figure}[p]\centering
  \hspace{-2mm}\includegraphics[clip,width=1.1\hsize]{img/positiondetection-new.png}
  \caption{最急降下法によって相対位置を解いたシミュレーション結果}\label{fig:relpo_s}
\end{figure}

端末 $i$ が測距できた他の端末の集合 $M(i)$ は後述する信号検出により決まる.


\section{信号検出}

\subsection{必要な同期制度のための信号検出の要件}

精密な測距・時刻同期のためには精密な信号検出が必要である.
まず許容される信号検出の誤差について考える,

スマートデバイスのサンプリング周波数を44100Hz
として
1サンプルあたりの時間解像度は約 1/44100 = 22.6μs
であるので,
1サンプルあたりの距離解像度は 22.6μs*340m/s=7.7mm(音速340m/sと仮定)
である.

人間の聴覚特性として,第一波面の法則という現象が知られている\cite{Haas}.
これは二つの音源からの音声が互いに50ms以上ずれると別の音源として知覚されるというものである.

この場合許容される誤差は
$\pm$ 50msの誤差におよそ $\pm$ 2205サンプル以内とかなり緩いものになる.
しかしこれでは距離誤差が $\pm$ 3.4m となり,教室空間での位置推定のパラメータとして使うにはとても大きなものになってしまう.
そのため,許容される誤差は音像定位ではなく距離測定の手法によって定められるといえる.
室内空間であれば,
$\pm$ 50cmの誤差つまり $\pm$ 64サンプル以内でパルスを同定できればよいとする.


\subsection{本研究で試したパルス圧縮の種類}

本研究で試行したパルス圧縮を表 \ref{tab:pulsecomps} にまとめる.

\begin{table}[p]\centering
  \caption{本研究で試したパルス圧縮の種類}
  \label{tab:pulsecomps}
  \begin{tabular}{ll|ccc}
    \hline
    パルス & 利点 & 欠点 \\
    \hline
    \parbox{10zw}{固定周波数パルス}
      & \parbox{10zw}{可聴域外のパルスに設定可能}
      & \parbox{10zw}{ノイズに弱い,ピーク発見が困難, \\ 可聴域外音声の発生・受信が安価な端末では困難}
      \\
    \hline
    \parbox{10zw}{全帯域チャープ信号}
      & \parbox{10zw}{パルス圧縮によりノイズに強い}
      & \parbox{10zw}{信号強度確保のために \\ 持続時間を長くするとピークが鈍くなる}
      \\
    \hline
    \parbox{10zw}{バーカー符号を用いた \\ 直接スペクトル拡散}
      & \parbox{10zw}{鋭いピークを持たせられる}
      & \parbox{10zw}{系列長が13までしかないため \\ SN比に限界がある}
      \\
    \hline
    \parbox{10zw}{バーカー符号化\\ チャープ信号}
      & \parbox{10zw}{短いチャープをバーカー符号化することで \\ チャープのピークを鋭く保ったまま13倍にできる}
      & \parbox{10zw}{高周波数帯の非線形歪みの影響で \\ 複数のピークが現れる}
      \\
    \hline
    \parbox{10zw}{M系列を用いた\\ 直接スペクトル拡散}
      & \parbox{10zw}{パルス持続時間を伸ばしてSN比を高めてもピークが鋭い}
      & \parbox{10zw}{持続時間を長くしすぎると非線形歪みの影響で \\ 複数のピークが現れる}
      \\
    \hline
  \end{tabular}
\end{table}

まず可聴域外に近い20000Hz前後の固定周波数パルスを用いた同期信号を試みた.
しかし,安価なスマートデバイスのスピーカとアンプとマイクロホンでは送受信ができない,
可聴域音にしても信号の検出が困難などの理由で不採用となった.

次に,チャープ信号を用いたパルス圧縮手法を試みた.
これにより信号強度を高める事ができた.
だが,信号のサイドローブが大きく鋭いピークを持たせられなかった.

チャープ信号に代わるもとして,バーカー符号を用いた直接スペクトル拡散を考えたが,
バーカー符号は系列長13までしかないので,そのまま直接拡散してもパルス持続時間が短くSN比を向上できない.
そこで,直接拡散において符号化する1波長を通常の正弦波ではなくひとつのチャープ信号にするパルス圧縮信号を試みた.
しかし,これも高周波数帯域の非線形歪みの影響を受け,複数のピークが現れた.

最後に,M系列符号を用いた直接スペクトル拡散を試みた.
これにより,ピークを鋭く保ったまま任意の持続時間のパルスを作ることができるようになった.
しかし,持続時間をむやみに長くすると,非線形歪みの影響を受けるようになった.

これらを順に説明する.

\subsection{固定周波数パルス}

本研究ではまず固定周波数パルスを試みた.
周波数を可聴域外に近い20000Hz前後にすることで,
利用者の耳には聞こえにくい同期信号を利用できると考えたからである.
これは実際MacBookProなどの高品質な端末では送受信可能だったが,
Nexus7(2013)などの比較的安価な端末のスピーカやマイクでは送受信できなかった.
また,この信号は可聴域としてもピークが鈍くで同期信号には適さなかった.

\subsection{チャープ信号}

そこで,チャープ信号を用いたパルス圧縮を試みた.
チャープ信号は時間に対して周波数を変化させた信号で,
同じ長さの固定周波数パルスに対してピークを鋭くすることができる利点がある.
しかし.持続時間を長くするとピークが鈍くなってしまう.
これによりマルチパスの影響により現れる複数のピークが重なってしまい,最初に届いた波のピークを特定することを困難とさせていた.
このため,この同期・測距信号には適さなかった.

\subsection{バーカー符号化チャープ信号}

そこでさらに,バーカー符号とチャープ信号のふたつのパルス圧縮方式を組み合わせる独自のパルス圧縮を考案した\cite{self_ac}.
バーカー符号とチャープ信号の二つのパルス圧縮技術を組み合わせて,チャープ信号をバーカー符号を用いてBPSKで変調したものである.
バーカー符号は直接スペクトル拡散方式で用いられる拡散系列で,
Diracのデルタ関数に近いするどい相関特性を持つ.
しかし,このようなパルスは系列長13までしか存在しないので,そのまま直接拡散しても持続時間を長くできずSN比を向上させられない.
そこで,バーカー符号を用いてBPSKして直接拡散するにあたって,チップ幅を1波長の正弦波ではなく,ひとつのチャープ信号を用いる事で,
チャープ信号の鋭さを保ったまま通常のチャープ信号の13倍のSN比を稼げると考えた
(図\ref{fig:barker_chirp},\ref{fig:barker_coded_chirp},\ref{fig:barker_coded_chirp_err},\ref{fig:corr}).

\begin{figure}[p]\centering
\includegraphics[clip,width=1.0\hsize]{img/barker_chirp.png}
\caption{チャープ信号を搬送波として長さ13のバーカー符号でBPSK変調信号したもの}\label{fig:barker_chirp}
\end{figure}

\begin{figure}[p]\centering
\includegraphics[clip,width=0.75\hsize]{img/barker_coded_chirp.png}
\caption{チャープ信号を搬送波として長さ13のバーカー符号でBPSK変調信号したもののスペクトルグラム(シミュレーション)}\label{fig:barker_coded_chirp}
\end{figure}


\begin{figure}[p]\centering
\includegraphics[clip,width=0.75\hsize]{img/barker_coded_chirp_err.png}
\caption{チャープ信号を搬送波として長さ13のバーカー符号でBPSK変調信号したもののスペクトルグラム
(MacBookAirによる実環境での自身の信号の計測結果)}\label{fig:barker_coded_chirp_err}
\end{figure}



\begin{figure}[p]\centering
\vspace{2mm}
\begin{small}
\includegraphics[clip,width=1.0\hsize]{img/rawdata.png}\\
A端末自身のパルス~ ~ ~ ~
B端末のパルス~ ~ ~ ~ ~
C端末のパルス\\\vspace{0.5mm}
\includegraphics[clip,width=0.8\hsize]{img/spectrogram.png}\hspace{1cm}\\
A端末自身のパルス~ ~ ~ ~ ~ ~ ~ ~ ~
B端末のパルス\\\vspace{0.5mm}
\includegraphics[clip,width=1.0\hsize]{img/corrA.png}\\
A端末のパルスに整合フィルタをかけたもの\\\vspace{0.5mm}
\includegraphics[clip,width=1.0\hsize]{img/corrB.png}\\
B端末のパルスに整合フィルタをかけたもの\\\vspace{0.5mm}
\includegraphics[clip,width=1.0\hsize]{img/corrC.png}\\
C端末のパルスに整合フィルタをかけたもの\\
\vspace{1mm}
\caption{A端末で観測したABC端末のパルス信号}\label{fig:corr}

\end{small}
\vspace{1mm}
\end{figure}

提案手法のパルス圧縮に関して,
以前利用していたこのバーカー符号化チャープ信号によるパルス圧縮ではなく,
M系列符号による直接拡散方式を用いた理由について以下で述べる.

本来barker codeはそのままサイン波に適用し直接拡散して使うものであるが,
系列長が最大で13までしかないため,パルス圧縮に上限があった.
一方でチャープ信号は引き伸ばせば引き伸ばすほどパルス圧縮されるが,サイドローブが大きくなるという問題を抱えていた.
そのため,この二つの手法を組み合わせて,短いup-chirpの繰り返しをbarker codeで拡散することで,
サイドローブを抑えながら圧縮する手法を試みた.
しかし,この手法は高周波成分において非線型歪みの影響を受けやすく,複数のピークが現れてしまうという問題を抱えていた.
%要図

一方で,M系列符号には系列長に制限がないため,そのような手法を必要とせず,直接拡散方式が利用できる.
%以前の実装は私がbarker codeによるパルス圧縮を知った時点で,まだM系列符号を知らなかった故のものであり,
%M系列符号が使える現在,そのようなハイブリット手法を使う理由はない.





\subsection{M系列による直接スペクトル拡散}

本研究ではM系列符号による直接スペクトル拡散方式によるパルス圧縮を用いた.
これはチャープ信号よりも非定常雑音に強い\cite{nonlinear}.
また,信号検出には通常の整合フィルタではなくピークを尖らせることができる,
フェイズオンリー整合フィルタ(phase-only matched filter: POF)\cite{pof, phaseonly2}を利用した(図\ref{fig:pof}).
フェイズオンリー整合フィルタは,整合フィルタに信号の周波数成分のみを利用することでサイドローブを抑えピークをとがらせることができるフィルタである.

\begin{figure}[p]\centering
  \hspace{-2mm}\includegraphics[clip,width=1.1\hsize]{img/POF.png}
  \caption{フェイズオンリー整合フィルタと整合フィルタ}\label{fig:pof}
\end{figure}

$$
\begin{aligned}
\mathrm{POF}[x_a, x_b]
&= \mathcal{F}^{-1}\left[\frac{\mathcal{F}\left[x_a(t)\right]^*}{|\mathcal{F}\left[x_a(t)\right]|}\mathcal{F}\left[x_b(t)\right]\right] \\
&= \mathcal{F}^{-1}\left[\frac{X_a^*(\omega)}{|X_a(\omega)|}X_b(\omega)\right]
\end{aligned}
$$


直接スペクトル拡散方式(direct sequence spread spectrum:DSSS)(図\ref{fig:DS})による測距システムの変復調方式を図\ref{fig:DME2}に示す.
搬送波には4410Hz正弦波を,拡散符号には符号長1023のM系列を用い,変調にはバイナリ位相シフトキーイング(BPSK)を使った.

\begin{figure}[p]\centering
  \hspace{-2mm}\includegraphics[clip,width=1.1\hsize]{img/DME2.png}
  \caption{変復調システム図}\label{fig:DME2}
\end{figure}

\begin{figure}[p]\centering
  \hspace{-2mm}\includegraphics[clip,width=1.1\hsize]{img/DSSS.png}
  \caption{直接スペクトラム拡散による信号変調過程}\label{fig:DSSS}
\end{figure}

この手法により,通常のパルスやチャープ信号,バーカー符号化チャープよりも鋭くSN比の高いピークが得られるようになった.
しかし,まだ問題は残っており,継続時間を長くしすぎると送受信側のシステムクロックのずれにより非線形歪みの影響を受け,ピークが複数現れるようになってしまう.
そこで



\subsection{伝送路推定用の参照信号を用いた信号同定}

復調した受信信号が雑音か有効な信号かを決定する処理を信号同定という.
伝送路における伝達関数 $H(\omega)$ において室内残響の影響
としてマルチパスによる影響を受けてしまう(図\ref{fig:multipath}).

\begin{figure}[p]\centering
  \hspace{-2mm}\includegraphics[clip,width=1.1\hsize]{img/multipath.png}
  \caption{反射波・回折波によるマルチパスの影響}\label{fig:multipath}
\end{figure}


そこで,同期パルスを測距用信号と,伝搬路を測定する参照波の二つに分離した.
参照波を用いた伝送路推定にはCDMA通信でのRake受信器において使われているサウンダ信号や,
OFDM通信などではパイロット信号が推定信号として使われる.
同期パルスから $n$ 秒後に参照信号を送り,
その二つの信号の相関を取ることで,
背景雑音とは別にパルス位置を特定することが可能になる(図\ref{fig:sounder}).

\begin{figure}[p]\centering
  \hspace{-2mm}\includegraphics[clip,width=1.1\hsize]{img/sounder.png}
  \caption{同期信号と参照信号}\label{fig:sounder}
\end{figure}

さらに参照信号と測距信号は互いに異なる同周期のM系列を用いた.
参照信号と測距信号を判別しやすくするためである.

この時刻が $n$ 秒ずれた二つの信号に対して時間窓で区切って相互相関をとることで,
信号が最も相関している区間,つまり信号の位置を特定することができる.
最後に,その区間相関値を閾値処理することで信号の到来を決定する.
今回は最大相関値前方での40\%の相関値を超えたピークを到来時刻としている.

相関関数の計算には Wiener-Khintchine の定理を使い,周波数領域での複素乗算としてFFTを使って計算することで計算量を減らすことができる.
長さの異なる信号の高速フーリエ変換には重畳加算法\cite{overwrap}を使った.


\begin{figure}[p]
  \centering
  \includegraphics[clip,width=1.05\hsize]{img/026.jpeg}
  \caption{変復調回路}\label{fig:henpuku}
\end{figure}


\begin{figure}[p]
  \centering
  \includegraphics[clip,width=1.05\hsize]{img/029.jpeg}
  \caption{信号同定過程}\label{fig:shousai}
\end{figure}




\clearpage


\section{システムクロック校正}

端末のシステムクロックの進みかたはハードウェアごとに微妙に異なる.
そのため,同期してから長時間経つと,次第に端末間で遅延が生じる.
このことは,同一音源を同期的に再生し続けると,次第にずれが聞き取れるようになってくることを意味する.
この遅延を検出する手法について述べる.

端末 A のクロックを $S_A$,
端末 B のクロックを $S_B$ して,
時刻 $t$ 後に
A のサンプル数が $i$ ,
B のサンプル数が $i+d$ だけ異なっているとする(図\ref{fig:phaseshift2}).
このとき,Aを基準とした遅延比率 $S_B/S_A$ を求めたい.

\begin{figure}[tb]\centering
  \hspace{-2mm}\includegraphics[clip,width=1.1\hsize]{img/phase_shift2.png}
  \caption{phaseshift2}\label{fig:phaseshift2}
\end{figure}

図より

$$\begin{aligned}
\frac{i}{S_A} &= \frac{i+d}{S_B} = t \\
\frac{S_B}{S_A} &= \frac{i+d}{i}
\end{aligned}$$

である.
遅延の検出においては,端末間の相対距離に変化がない限り,
端末Aが端末Bへとパルスを発生するだけで良く,
端末BはAへと返答パルスを返す必要はない.
既に同期済みでありAからBへの伝搬時間は算出済みだからである.



\section{音圧校正}
スマートデバイス毎にマイクロホンやスピーカのアンプ出力は異なるため,そのままではDBAP法は使えない.
ここでは,機器ごとの音圧を校正する手法を示す.

LTI(線形時不変)システムを仮定する.
$N$ 台の端末の番号を $i,j \in \{1\dots N\}$ とする.
端末 $i$ から端末 $j$ への信号伝達を考える.
$e$ を端末 $i$ で生成した単位振幅,
$v_i$ を端末 $i$ のスピーカアンプの増幅係数,
$m_j$ を端末 $j$ のマイクロホンアンプの増幅係数,
$d_{ij}=d_{ij}$ を $ij$ 間の測定距離,
$x_{ij}$ を $j$ が観測した $i$ からの信号の振幅とする.
音波の振幅は距離に反比例して減衰することが知られているので,
音声信号の伝達は

\begin{figure}[tb]\centering
  \hspace{-2mm}\includegraphics[clip,width=1.1\hsize]{img/sound_pressure_calibration.png}
  \caption{soundpressurecalibration}\label{fig:soundpressurecalibration}
\end{figure}

$$
e v_i \frac{1}{d_{ij}} m_j = x_{ij}
$$

とモデル化できる(図\ref{fig:soundpressurecalibration}).
このとき,ある端末 $k$ の出力係数 $v_k$ と他の端末 $i$ の出力係数 $v_i$ との比 $v_i/v_k$ を求めたい.

$$
e v_i m_j = x_{ij}d_{ij}
$$

なので

$$\begin{aligned}
\frac{e v_j m_j}{e v_k m_j} &= \frac{x_{ij} d_{ij}}{x_{kj} d_{kj}} \\
\frac{v_j}{v_k} &= \frac{x_{ij} d_{ij}}{x_{kj}d_{kj}} \\
\end{aligned}$$

である.
観測した振幅 $x_{ij}$ および測定距離 $d_{ij}$ は誤差を含むので,
それらを平均した $\hat{v_i}$ は

$$
\frac{\hat{v_i}}{v_k} = \frac{1}{N} \sum_{i\neq j \neq k} \frac{x_{ij} d_{kj}}{x_{kj}d_{ij}} \\
$$

と定義できる.$d_{ii}$ のときは距離が0となりゼロ除算が発生するので,
$i\neq j \neq k$ としている.

一番出力の低い端末 $k$ の出力係数 $v_k$ を基準とすることで,
すべての端末において定格出力を守ることができる.

\section{隣接ノードでない端末間の同期,音圧校正,クロック校正}


互いに互いの信号を検出できなかったノード間での校正を考える.
図\ref{fig:networktopology}に隣接ノードではない端末を含むネットワークを示す.

\begin{figure}[tb]\centering
  \hspace{-2mm}\includegraphics[clip,width=1.1\hsize]{img/network_topology.png}
  \caption{networktopology}\label{fig:networktopology}
\end{figure}

このとき端末 A と端末 C 間では同期・測距ができていないが,
互いに端末 B とは同期・測距できているという状況である.

\subsection{同期}

時刻の基準となる
端末 $A$ がパルスを発した時間を $t_{AA}$ として
そのパルスが端末 $B$ に届いた時間は $t_{AB}$ とする.
また,
端末 $B$ がパルスを発した時間を $t_{AB}$ として
そのパルスが端末 $C$ に届いた時間は $t_{BC}$ とする.
そしてそれぞれの伝達時間を $d$ とすると図\ref{fig:reldelay}のようになる.

\begin{figure}[tb]\centering
  \hspace{-2mm}\includegraphics[clip,width=1.1\hsize]{img/rel_delay.png}
  \caption{reldelay}\label{fig:reldelay}
\end{figure}

このとき,まず端末 $C$ は端末 $B$ と同期して,
その後端末 $B$ と端末 $A$ の時刻ずれ情報をもとにさらに端末 $A$ との同期ができる.


\subsection{音圧校正}

端末 $A$ を基準に音圧校正を考えると

$$
\frac{v_B}{v_A} = \frac{x_{iB} d_{iB}}{x_{AB}d_{AB}} \\
\frac{v_C}{v_B} = \frac{x_{iC} d_{iC}}{x_{BC}d_{BC}} \\
$$

であるので

$$
\frac{v_C}{v_A} =
\frac{v_B v_C}{v_A v_B} =
\frac{x_{iB} x_{iC} d_{iB} d_{iC}}{x_{AB} x_{BC} d_{AB} d_{BC}}
$$

とすれば端末 A と端末 C の出力比率を求められる.

\subsection{クロック校正}


\section{同期・測距・校正手法のための制御システム}
同期のためには複数の端末がパルスを出し合わなければならないが,
いつどの端末がパルスを出すのか\ref{fig:TDMA},といったスケジューリングをどうするかについて述べる.

\begin{figure}[p]\centering
  \hspace{-2mm}\includegraphics[clip,width=1.1\hsize]{img/TDMA.png}
  \caption{TDMA}\label{fig:TDMA}
\end{figure}

図\ref{fig:network2}にシステム全体のネットワーク構成を示す.

\begin{figure}[p]\centering
  \hspace{-2mm}\includegraphics[clip,width=1.1\hsize]{img/network2.png}
  \caption{ネットワーク構成}\label{fig:network2}
\end{figure}

基本的には端末間の通信を中継するリレーサーバを中心としたスター型ネットワークである.
また,スピーカアレイに参加しない特別なノードとして,
計算用ノードと仮想音源を設定する制御用ノードがある.

次の図\ref{fig:flowchart}に同期・測距するまでのシステムフローを示す.

\begin{figure}[p]\centering
  \hspace{-2mm}\includegraphics[clip,width=1.1\hsize]{img/flowchart.png}
  \caption{同期制御フロー}\label{fig:flowchart}
\end{figure}

今回の実装では中継サーバが同期アルゴリズムを制御している.
すべてのコマンドはリクエスト-レスポンスで成り立っており,リクエストを受けた端末は必ずレスポンスを返さねばならない.
まず,中継サーバはスピーカアレイを構成する端末に対して ping コマンドを送信し,
アレイに参加できる端末を確認する.
次に,全端末に対して録音をするように beginRec コマンドを送信する.
そして,各端末の放つパルスが排他的になるように,
パルスを放つ端末ごとにstartPulse,beepPulse,stopPulseコマンドを繰り返し送信する.
startPulseとstopPulseコマンドは,
この時間区間内にいずれかの端末からパルスが発信されることを示すもので,
後にパルス位置を検出するときの計算量を減らすためのコマンドである.
beepPulseは任意の一台の端末に対して,パルスを送信するように促すコマンドである.

そしてすべての端末が互いに排他的にパルス発生し終えると,
最後にstopRecという録音終了コマンドを送信する.
その後,collectコマンドで各端末が録音したデータを集計し,
計算用サーバへ送信する.
計算用サーバは,それぞれの端末間のパルスの受信時刻を
先述の手法で検出し,相対信号伝達時間と相対距離計測,空間配置推定する.
その後,それらの情報を中継サーバを介して制御用端末へ送信する.

\section{仮想音源配置によるDBAPアレイスピーカ制御システムUI}

仮想音源を配置し制御するための端末のユーザインターフェースを示す(図\ref{fig:relpos}).

\begin{figure}[p]\centering
  \hspace{-2mm}\includegraphics[clip,width=1.1\hsize]{img/relpos.png}
  \caption{仮想音源配置UI}\label{fig:relpos}
\end{figure}

図のように推定した端末の分布図と,仮想音源を表示する.
仮想音源VSをドラッグすることで,DBAP法によって出力する振幅を計算し,各端末へ振幅を配信することで音像定位する.
当然,音を鳴らしながら音源を移動させることも可能である.


\section{P2Pリングネットワークを使った計算量の削減}
以前は
しかしながらパルス同定アルゴリズムのデバッグのしやすさの観点から、各端末に計算を分散して並列化による計算時間を削減するよりも、一台のパソコンで処理したほうが結果が比べやすいという開発上の観点から計算サーバーを導入しています
また、以前計算が停止した原因は私の書いたプログラムにゼロ除算を含む計算が生じる可能性があたったためで、nexus7でも計算自体は可能です。

