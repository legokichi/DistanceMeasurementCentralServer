\chapter{関連研究}
%または
%\chapter{菌類}

\section{位置情報に基づく情報提供}

複数のスマートデバイスで複数のユーザにサービスを提供するシステムとして,
位置情報を利用した携帯端末への音声情報配信\cite{kawagoe14}がある.
このようなサービスは,
ユーザ情報を取得し共有するための
個人を対象としたサービスであり,集団を対象とはせず,
端末間の通信で実空間に刺激を形成するものでもない.


\section{マルチチャンネルスピーカ}

これまでの多チャンネルスピーカによる音場再現手法としては,
波面合成法(WFS: wave field synthesis)\cite{wfs, 木村敏幸},
高次アンビソニックス法(HOA: higher order Ambisonics)\cite{hoa, 小山翔一},
境界音場制御法\cite{sfc, 伊勢史郎, 岡田耕介},
などが知られている\cite{鈴木陽一, 濱崎公男, 尾本章}.
また,パラメトリックスピーカを用いて特定の場所に音像を定位する手法\cite{paramsp}がある.
他にも,複数のスピーカを用いて振幅パンニングをするVector-base Amplitude Panning(VBAP)法\cite{PULKKI}や
Distance-based Amplitude Panning(DBAP)\cite{dbap}がある.
さらに,ユーザが移動しても音像を提供することができる手法\cite{湯山雄太}なども開発されている.
しかしながら,これらの手法はどれも特殊な機器と特別な設定が必要であり,
公共空間への導入が困難である.


\section{アドホックマイクロホンアレイ}

これまで,端末間同期手法に関して,
音の発信を利用したスマートフォンアレイの機器位置推定\cite{shibata13}や
音の発信を利用したキャリブレーションに基づくアドホックマイクロホンアレイによる音源位置推定\cite{shibata14}がある.
マイクロホンアレイは複数スマートデバイスのマイクロホンで取得した多チャネル信号を処理し,音源位置推定,音源分離などを行う\cite{小野順貴14-1, 小野順貴14-2}.
これはスマートデバイスでアレイ処理をする点においては似ているが,
本研究ではスピーカアレイを構築するために相対位置推定や同期を行うためのマイクロフォンの利用という点で異なる.


\section{パルス圧縮手法}

また,先の手法は音による位置測定における測距パルスに
時間引き伸ばしパルス(Time Stretched Pulse:TSP) を使用している.
これは音響測定におけるインパルス応答を測定するための信号であり\cite{Aoshima},
継続時間の長いTSPは相関結果のサイドローブが大きくなるため距離を測定するための信号としては不適当であるという問題がある.


\section{本研究のスタンス}

以上をまとめると,
スマートデバイスを用いた情報提供システムには,個人向けの研究が目立つ.
また,複数端末を用いてマイクロホンアレイを構築する研究は存在するが,複数端末を用いてのスピーカアレイを構築する手法は比較的未開拓分野と言える.
そして,複数のスマートデバイスを用いて実空間内の複数の人間に働きかける,という本研究のシステムは,
今後さらに生活空間にスマートデバイスが普及することを考えると,
パラコミュニケーションを実現する手段としても重要であると言える.


%\section{菌類とは}
%\section{細菌類とは}
%\section{納豆菌とは}
%\section{納豆菌の既知の効能}
%\section{本研究のスタンス}
