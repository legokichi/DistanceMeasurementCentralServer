\chapter{結論}
\section{本研究の成果}
%と,限定された点を明らかにしたり,
%さらに改善されるべき点を述べる

本研究では,広い空間内にいる多人数の端末を用いた音声コミュニケーション手法として,
音像定位を用いた同時多発的なやり取り実現のために,
参加者個人のスマートデバイスを用いてスピーカアレイを構築するシステムを提案した.
スマートデバイスを用いてスピーカアレイを構築するためのシステムの要件として,
DBAP法を用いた音像定位,パルス往復による端末間同期および端末間測距,
端末間相対距離を元にした最急降下法みによる相対位置推定,
環境雑音に強いパルス圧縮手法とその信号処理方法,
異種端末間でのパルス交換を用いた音圧校正およびシステムクロック校正,
インターネットを用いたスピーカアレイ制御手法などを検討し,
実装した.
そして,計測実験および聴衆実験を行い,
スマートデバイスを用いて,DBAP法を用いたスピーカアレイを構築できることを示した.

\section{本研究の展望}
%どうしたらもっとよくなるか
%どうしたら残ってる問題を解決できそうか.

現在残っている問題点として,
音圧校正手法の妥当性,
多端末での各種校正手法の妥当性,
マルチパス環境下,ダイレクトパスのない環境下での信号処理の問題などがある.
これらの解決策をひとつひとつ検証し実装していくことで,
提案システムはより実用に耐えうるものになると言える.


\section{本研究の総括と結論}

教室空間における複数の個人所有のスマートデバイスを用いてスピーカアレイを構築し,DBAP法を用いた音像定位するシステムを提案した.
各端末を音声パルスを用いることで時刻同期し,位置を推定し,仮想音源を配置できることを示した.
これにより教室空間における新しい音声コミュニケーション手法を実現した.
