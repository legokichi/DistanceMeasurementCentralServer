\begin{thebibliography}{99}
%\begin{bibliography}{}
\addcontentsline{toc}{chapter}{参考文献}

\bibitem{tpsn}      GANERIWAL, Saurabh, KUMAR, Ram, SRIVASTAVA, Mani B. Timing-sync protocol for sensor networks. In: Proceedings of the 1st international conference on Embedded networked sensor systems. ACM, 2003. p. 138-149.

% intro
\bibitem{11ubi}       Yu Zheng, Yanchi Liu, Jing Yuan, Xing Xie. Urban computing with taxicabs. In: Proceedings of the 13th international conference on Ubiquitous computing. ACM, 2011. p. 89-98.
\bibitem{辻賢太郎}      辻賢太郎, 上岡英史. センサ情報に基づいたユーザへのアラーティング方式 (ホームネットワーク, ユビキタスネットワーク, コンテキストアウェア, e コマース及び一般). 電子情報通信学会技術研究報告. MoMuC, モバイルマルチメディア通信, 2008, 108.290: 15-20.
\bibitem{surechigai}  末廣 創,佐藤 文明,すれ違い通信による情報伝搬モデルの特性評価,情報処理学会全国大会講演論文集 第72回(ネットワーク), 155-156, 2010-03-08, 2010.
\bibitem{clicker}     TREES, April R., JACKSON, Michele H. The learning environment in clicker classrooms: student processes of learning and involvement in large university‐level courses using student response systems. Learning, Media and Technology, 2007, 32.1: 21-40.
\bibitem{imakiku}     株式会社天問堂, Imakiku, \url{http://tenmondo.com/products/imakiku/index.html}  (2015時点)
\bibitem{twitter}     後藤真孝. WISS 2010 と WISS 2011 での改革. コンピュータ ソフトウェア, 2012, 29.4: 4\_3-4\_8.

% related
\bibitem{kawagoe14} 河越嵩介, 田中二郎, 神場知成. 位置情報を利用した携帯端末への音声情報配信. 情報処理学会第 76 回全国大会, 2014, 4: 2.
\bibitem{wfs}       BERKHOUT, Augustinus J., DE VRIES, Diemer, VOGEL, Peter. Acoustic control by wave field synthesis. The Journal of the Acoustical Society of America, 1993, 93.5: 2764-2778.
\bibitem{木村敏幸}   木村敏幸. 波面合成技術の研究動向 (< 小特集> 音場再生技術の研究動向). 日本音響学会誌, 2011, 67.11: 538-543.
\bibitem{hoa}       DANIEL, Jérôme. Spatial sound encoding including near field effect: Introducing distance coding filters and a viable, new ambisonic format. In: Audio Engineering Society Conference: 23rd International Conference: Signal Processing in Audio Recording and Reproduction. Audio Engineering Society, 2003.
\bibitem{小山翔一}   小山翔一. 音場再現技術における数理問題: 波面合成・高次アンビソニックスの数理 (< 小特集> 近年の音響信号処理における数理科学の進展). 日本音響学会誌, 2012, 68.11: 584-589.
\bibitem{sfc}       ISE, Shiro. A principle of sound field control based on the Kirchhoff-Helmholtz integral equation and the theory of inverse systems. Acta Acustica united with Acustica, 1999, 85.1: 78-87.
\bibitem{伊勢史郎}   伊勢史郎. 境界音場制御 (< 小特集> 音場再生技術の研究動向). 日本音響学会誌, 2011, 67.11: 532-537.
\bibitem{岡田耕介} 岡田耕介, 川口孝幸, 榎本成悟, 伊勢史郎. コンパクトな没入型聴覚ディスプレイの試作と評価. 日本音響学会誌, 2005, 62.1: 32-41.
\bibitem{鈴木陽一} 鈴木陽一, 西村竜一. 3-1 超臨場感音響の展開 (3. 超臨場感音響技術,< 特集> 超臨場感コミュニケーションの近未来像). 電子情報通信学会誌, 2010, 93.5: 392-396.
\bibitem{濱崎公男} 濱崎公男. マルチチャンネル音響 (< 小特集> 音場再生技術の研究動向). 日本音響学会誌, 2011, 67.11: 526-531.
\bibitem{尾本章} 尾本章. 音場再生技術について (< 小特集> 音場再生技術の研究動向). 日本音響学会誌, 2011, 67.11: 520-525.
\bibitem{paramsp}   青木茂明,清水一博,伊藤昂輝.パラメトリックスピーカを用いた再生時の音像定位.信学技報 EA研究会,vol.114,no.423, pp.33--38, 2015.
\bibitem{PULKKI} PULKKI, Ville. Virtual sound source positioning using vector base amplitude panning. Journal of the Audio Engineering Society, 1997, 45.6: 456-466.
\bibitem{dbap}      Trond Lossius, Pascal Baltazar, and Theo de la Hogue. DBAP–distance-based amplitude panning. Proc of ICMC 2009, pp.489--492, 2009.
\bibitem{湯山雄太} 湯山雄太, 宮部滋樹, 猿渡洋, 鹿野清宏. スイートスポット以外で複数音源の方位を提示可能なバイノーラル再現 (立体音響・トランスデューサ/一般). 電子情報通信学会技術研究報告. EA, 応用音響, 2007, 107.370: 49-54.

\bibitem{HENNECKE} HENNECKE, Marius H., FINK, Gernot A. Towards acoustic self-localization of ad hoc smartphone arrays. In: Hands-free Speech Communication and Microphone Arrays (HSCMA), 2011 Joint Workshop on. IEEE, 2011. p. 127-132.
\bibitem{小野順貴14-1} 小野順貴, 宮部滋樹, 牧野昭二. 非同期分散マイクロホンアレイに基づく音響信号処理. 日本音響学会誌, 2014, 70.7: 391-396.
\bibitem{小野順貴14-2} 小野順貴, LET Kien, 宮部滋樹, 牧野昭二. アドホックマイクロホンアレー. 電子情報通信学会 基礎・境界ソサイエティ Fundamentals Review, 2014, 7.4: 336-347.
\bibitem{shibata13} 柴田一暁, 小野順貴, 亀岡弘和, 音の発信を利用したスマートフォンアレイの機器位置推定. 音講論集, pp. 591–592, Sept. 2013.
\bibitem{shibata14} 柴田一暁, 小野順貴, 亀岡弘和. 音の発信を利用したキャリブレーションに基づくアドホックマイクロホンアレイによる音源定位. 音講論集, March, 2014.
\bibitem{Aoshima}   N. Aoshima, Computer-generated pulse signal applied for sound measurement, J. Acoust. Soc. Am., vol.69, no.5,1484--1488, 1981.

\bibitem{SYED} AA Syed, JS Heidemann. Time Synchronization for High Latency Acoustic Networks. In: INFOCOM. 2006.
\bibitem{XU} B Xu, G Sun, R Yu, Z Yang. High-accuracy TDOA-based localization without time synchronization. Parallel and Distributed Systems, IEEE Transactions on, 2013, 24.8: 1567-1576.
\bibitem{LU} LU, Chao, WANG, Shuo, TAN, Min. A time synchronization method for underwater wireless sensor networks. In: Control and Decision Conference, 2009. CCDC'09. Chinese. IEEE, 2009. p. 4305-4310.
\bibitem{AKYILDIZ} AKYILDIZ, Ian F., POMPILI, Dario, MELODIA, Tommaso. Underwater acoustic sensor networks: research challenges. Ad hoc networks, 2005, 3.3: 257-279.
\bibitem{浜田龍平} 浜田龍平, 村田佳洋. 水中ワイヤレスセンサネットワークにおける伝搬遅延を考慮したタイムスロットスケジューリング. 情報処理学会研究報告. MPS, 数理モデル化と問題解決研究報告, 2012, 2012.1: 1-6.
\bibitem{LAZIK} P Lazik, N Rajagopal, B Sinopoli. Ultrasonic time synchronization and ranging on smartphones. In: Real-Time and Embedded Technology and Applications Symposium (RTAS), 2015 IEEE. IEEE, 2015. p. 108-118.
\bibitem{PENG} C Peng, G Shen, Y Zhang, Y Li, K Tan. Beepbeep: a high accuracy acoustic ranging system using cots mobile devices. In: Proceedings of the 5th international conference on Embedded networked sensor systems. ACM, 2007. p. 1-14.
\bibitem{ENS} Alexander Ens,1 Fabian Höflinger,1 Johannes Wendeberg, Joachim Hoppe, Rui Zhang, Amir Bannoura, Leonhard M. Reindl, and Christian Schindelhauer. Acoustic Self-calibrating System for Indoor Smart phone Tracking (ASSIST). International Journal of Navigation and Observation, 2015.694695: 1-14.
\bibitem{JANSON} JANSON, Thomas, SCHINDELHAUER, Christian, WENDEBERG, Johannes. Self-localization application for iPhone using only ambient sound signals. In: Indoor Positioning and Indoor Navigation (IPIN), 2010 International Conference on. IEEE, 2010. p. 1-10.

\bibitem{seigoufilter} 滑川俊彦,奥井重彦,衣斐信介,通信方式(第二版),森北出版,2012.
\bibitem{pulsecompress} 近藤倫正, 実森彰郎,大橋由昌,計測・センサにおけるディジタル信号処理,昭晃堂,1993.
\bibitem{レーダ技術} レーダ技術 電子情報通信学会 編 電子情報通信学会 1996
\bibitem{レーダ信号処理技術} レーダ信号処理技術 電子情報通信学会 編著 電子情報通信学会 1991
\bibitem{稲葉敬之11} 稲葉敬之, 桐本哲郎. 車載用ミリ波レーダ. 自動車技術, 2010, 64.2: 74-79.
\bibitem{山口功} 山口功. 合成開口ソーナ. 日本音響学会誌, 2003, 59.12: 723-728.
\bibitem{acoima} アコースティックイメージング / 秋山いわき 編著,蜂屋弘之, 坂本慎一 共著 コロナ社 2010 (音響テクノロジーシリーズ , 15)
\bibitem{海洋音響の基礎と応用} 海洋音響の基礎と応用 海洋音響学会 編 成山堂書店 2004
\bibitem{水中音響学} 水中音響学 Robert J.Urick 著,三好章夫 訳,新家富雄 監修 京都通信社 2013
\bibitem{高野忠01} 宇宙通信および衛星放送 高野忠 [ほか]共著 コロナ社 2001 (宇宙工学シリーズ , 4)
\bibitem{高野忠00} 宇宙における電波計測と電波航法 高野忠 [ほか]共著 コロナ社 2000 (宇宙工学シリーズ , 1)
\bibitem{電子戦の技術基礎編} 電子戦の技術 : A First Course in Electronic Warfare 基礎編 デビッド・アダミー 著,河東晴子, 小林正明, 阪上廣治, 徳丸義博 訳 東京電機大学出版局 2013
\bibitem{電子戦の技術拡充編} 電子戦の技術 = A Second Course in Electronic Warfare 拡充編 デビッド・アダミー 著,河東晴子, 小林正明, 阪上廣治, 徳丸義博 訳 東京電機大学出版局 2014
\bibitem{電子戦の技術通信電子戦編} 電子戦の技術 通信電子戦編 デビッド・アダミー 著,河東晴子, 小林正明, 阪上廣治, 徳丸義博 訳 東京電機大学出版局 2015
\bibitem{谷本正幸} 谷本正幸, 住吉浩次, 駒井又二. 変形 M 系列を用いた同期式スペクトル拡散多重通信方式. 電子情報通信学会論文誌 B, 1984, 67.3: 297-304.
\bibitem{specto} スペクトル拡散通信とその応用 / 丸林元, 中川正雄, 河野隆二 共著,電子情報通信学会 編 電子情報通信学会 1998
\bibitem{yamauchi} スペクトラム拡散通信 : 次世代高性能通信に向けて / 山内雪路 著 東京電機大学出版局 1994
\bibitem{Dixon} 最新スペクトラム拡散通信方式 / R.C.Dixon 著,立野敏 [ほか]訳 日本技術経済センター 197
\bibitem{渋澤功} 渋澤功, 金田豊. 実環境雑音下におけるインパルス応答測定波形の最適切り出し方法の検討. 電子情報通信学会技術研究報告. EA, 応用音響, 2012, 112.292: 51-56.
\bibitem{金田豊} 金田豊. インパルス応答測定信号と測定誤差. 日本音響学会誌, 2013, 69.10: 549-554.
\bibitem{守谷直也} 守谷直也, 金田豊. 雑音に起因する誤差を最小化するインパルス応答測定信号. 日本音響学会誌, 2008, 64.12: 695-701.


\bibitem{dsn} O’DEA, Andrew, KINMAN, Peter. Pseudo-Noise and Regenerative Ranging. 2015.
\bibitem{tomography} 竹内倶佳. 海洋音響トモグラフィ (< 小特集> 水中音響). 日本音響学会誌, 1986, 42.7: 575-585.

% system
\bibitem{nonlinear} 金田豊. M 系列を用いたインパルス応答測定における誤差の実験的検討. 日本音響学会誌, 1996, 52.10: 752-759.
\bibitem{self_ac} 伊納洋祐, 吉田侑矢, 米澤朋子,	複数端末の音響的位置推定と同期による空間音響環境構築システムの提案,	ASJ 2014 autumn,	pp.1439--1440,	2014.
\bibitem{pof}       HORNER, Joseph L., GIANINO, Peter D. Phase-only matched filtering. Applied optics, 1984, 23.6: 812-816.
\bibitem{phaseonly2} 原理がわかる・現場で使える信号処理 / 伊東一良 編,浅野晃, 津村徳道, 野村孝徳, 廣川勝久, 的場修 著 丸善 2009
\bibitem{overwrap}  RABINER, Lawrence R., GOLD, Bernard. Theory and application of digital signal processing. Englewood Cliffs, NJ, Prentice-Hall, Inc., 1975. 777 p., 1975, 1.
\bibitem{Haas}      HAAS, Helmut. The influence of a single echo on the audibility of speech. Journal of the Audio Engineering Society, 1972, 20.2: 146-159.


% exp
\bibitem{ku-kanonkyo} 空間音響学 / 飯田一博, 森本政之 編著,福留公利, 三好正人, 宇佐川毅 共著 コロナ社 2010 (音響サイエンスシリーズ , 2)
\bibitem{onbasaigen} 音場再現 = Sound Field Reproduction / 安藤彰男 著 コロナ社 2014 (音響サイエンスシリーズ , 10)
\bibitem{onkyoukougaku} 音響工学基礎論 = Fundamentals of Engineering Acoustics / 飯田一博 著 コロナ社 2012
\bibitem{森本政之09} 森本政之. 音場の拡散と音像の空間印象について (< 小特集> 室内音響における拡散研究の最新動向). 日本音響学会誌, 2009, 65.11: 584-588.
\bibitem{森本政之90} 森本政之, 藤森久嘉, 前川純一. みかけの音源の幅と音に包まれた感じの差異. 日本音響学会誌, 1990, 46.6:	449-457.
\bibitem{森本政之93} 森本政之, 飯田一博. みかけの音源の幅と第 1 波面の法則の関係. 日本音響学会誌, 1993, 49.2: 84-89.
\bibitem{崔瑛芝} 崔瑛芝, 比嘉規晶, 古屋浩, 藤本一寿. 後期音の到来方向が 「音に包まれた感じ」 に与える影響. 学術講演梗概集. D-1, 環境工学 I, 室内音響・音環境, 騒音・固体音, 環境振動, 光・色, 給排水・水環境, 都市設備・環境管理, 環境心理生理, 環境設計, 電磁環境, 1999, 1999: 35-36.
\bibitem{上杉信敏} 上杉信敏, 金田豊. 音源方向推定に及ぼす室内反射音影響の分析的検討 (音響・超音波サブソサイエティ合同研究会). 電子情報通信学会技術研究報告. EA, 応用音響, 2007, 106.484: 7-12.
\bibitem{田中雅史} 田中雅史, 金田豊, 小島順治. 音源方向推定法の室内残響下での性能評価. 日本音響学会誌, 1994, 50.7: 540-548.

% discus
\bibitem{morimoto95} Morimoto, Masayuki, and Kazuhiro Iida. A practical evaluation method of auditory source width in concert halls. Journal of the Acoustical Society of Japan (E) vol.16, no.2, pp.59--69, 1995.
\bibitem{barron81} M. Barron, and A. H. Marshall. Spatial impression due to early lateral reflections in concert halls: the derivation of a physical measure. Journal of Sound and Vibration, vol.77, no.2, pp.211--232, 1981.
\bibitem{suehiro06} 末廣大地, 翁長博, 池田哲朗. 音楽ホールにおける音に包まれた感じに対応する物理指標の検討. 日本建築学会環境系論文集,no.599, pp.1--7, 2006.

\bibitem{和泉洋一} 和泉洋一, 高橋弘太, 岩倉博. A-10-1 特性にばらつきのあるマイクロホンアレイによる音源抽出. 電子情報通信学会ソサイエティ大会講演論文集, 2000, 2000: 178.
\bibitem{澤上佳希} 澤上佳希, 岩井将行, 瀬崎薫. 異種スマートフォン間の音圧校正手法の提案 (モバイルユビキタス/センサ技術, アドホックネットワーク, RFID, 一般及び技術展示). 電子情報通信学会技術研究報告. USN, ユビキタス・センサネットワーク, 2012, 111.386: 111-116.

\bibitem{合原一究} 合原一究, 武田龍, 水本武志, 高橋徹, 奥乃博. ニホンアマガエルの同期した発声行動に関する数理的研究および音響信号解析 (第 5 回生物数学の理論とその応用). 2009.
\bibitem{谷口義明} 谷口義明, 若宮直紀, 村田正幸. パルス結合振動子モデルにおける進行波状態を応用したセンサネットワークのための自己組織型通信機構. 電子情報通信学会技術研究報告. NS, ネットワークシステム, 2006, 106.167: 17-20.
\bibitem{木村宏人} 木村宏人, 江口正道. パルス振動子を内蔵し同期発光するマイコンホタルの製作. 山形県立産業技術短期大学校紀要= Research report of Yamagata College of Industry \& Technology, 2002, 8: 34-37.

\bibitem{chordalg}I. Stoica, R. Morris, D. Karger, M. F. Kaashoek, H. Balakrishnan.Chord: A scalable peer-to-peer lookup service for internet applications.ACM SIGCOMM Computer Communication Revies, vol. 31, issue 4, pp.149--160,2001.
\bibitem{SHAKER} SHAKER, Ayman, REEVES, Douglas S. Self-stabilizing structured ring topology p2p systems. In: Peer-to-Peer Computing, 2005. P2P 2005. Fifth IEEE International Conference on. IEEE, 2005. p. 39-46.

% 結合振動子
\begin{comment}
\bibitem{小松崎俊彦} 小松崎俊彦, 佐藤秀紀, 岩田佳雄. 502 CA によるホタル群集の発光同期シミュレーション (OS 5-1 振動・制御問題)(オーガナイズドセッション 5: 機械の動的問題). 講演論文集, 2001, 2001.38: 135-136.
\bibitem{鯰江一也} 鯰江一也, 栗田裕, 松村雄一. 723 ホタルの集団同期発光を模擬した相互引込みの実現. In: Dynamics \& Design Conference. 一般社団法人日本機械学会, 2007.
\bibitem{伊藤大輔} 伊藤大輔, et al. 周期外力を加えた電子ホタルにおける分岐と同期現象. 電子情報通信学会技術研究報告. NLP, 非線形問題, 2012, 111.395: 81-86.
\end{comment}


% Mseq+chirp
% \bibitem{佐藤友治} 佐藤友治, et al. M 系列符号を用いた超音波距離計測におけるパルス圧縮の多チャンネル化. 日本音響学会 2008 年秋季研究発表会 講演論文集, 2008, 1527-1528.





% 同期
\begin{comment}
\bibitem{米川賢治} 米川賢治, et al. 無線センサネットワークにおける低消費電力な時刻推測手法 (ユビキタス・センサネットワークの要素技術, コンテクストの抽出, スマートスペース, ユビキタス生活支援, 一般). 電子情報通信学会技術研究報告. USN, ユビキタス・センサネットワーク, 2010, 110.130: 41-46.
\end{comment}

% 測距
\begin{comment}
\bibitem{kokuhayashi} 航行援助無線 / 林良治 著 近代科学社 1961
\bibitem{大槻知明} 大槻知明. 位置推定技術. 信学技報, 2009, 1-5.

\bibitem{岩谷晶子} 岩谷晶子, 西尾信彦, 徳田英幸. GOMASHIO: アドホックセンサネットワークにおけるノード位置特定方式. 情報処理学会, モバイルコンピューティングとワイヤレス通信研究会, 2001, 2001.108: 22-30.
\bibitem{茂出木敏雄08} 茂出木敏雄, et al. 音楽電子透かし技術を用いたモバイル端末位置情報の検出手法. 情報処理学会研究報告モバイルコンピューティングとユビキタス通信 (MBL), 2008, 2008.94 (2008-MBL-046): 25-32.
\bibitem{西村康孝} 西村康孝, 田坂和之, 吉原貴仁. 音波を使った携帯通信端末間の方向推定方式. 電子情報通信学会論文誌 B, 2012, 95.11: 1404-1413.
\bibitem{伊東正安} 医用音響工学 / 伊東正安, 望月剛 著 東京電機大学出版局 2014
\bibitem{komatsu2014} 小松潤也, 塩田茂雄, センサ協調位置推定: 相互多辺測量法と多次元尺度構成法の精度比較, 信学技報, ASN 114(65), pp.127--132, 2014.
\end{comment}




% pulse complession
\begin{comment}
\bibitem{永尾典子} 永尾典子, et al. Barker 符号を用いた近距離地中レーダ. 電子情報通信学会技術研究報告. SANE, 宇宙・航行エレクトロニクス, 2001, 101.529: 79-84.
\bibitem{富澤良行} 富澤良行, 新井郁男. BARKER 符号を用いたパルス圧縮地中レーダ. 電子情報通信学会総合大会講演論文集, 1997, 1997.1: 226.
\bibitem{稲葉敬之04} 稲葉敬之. Forward/Backward Barker 符号を用いたレーダ信号 (リモートセンシング及び一般). 電子情報通信学会技術研究報告. SANE, 宇宙・航行エレクトロニクス, 2004, 104.97: 37-42.
\bibitem{金基重} 金基重, 佐藤拓朗. B-8-37 チャープ信号によるアダプティブアレーアンテナ技術研究 (B-8. 通信方式, 一般セッション). 電子情報通信学会総合大会講演論文集, 2010, 2010.2: 300.
\bibitem{大久保寛} 大久保寛, 藤原幹己, 田川憲男. FM チャープパルス圧縮を用いた超解像法による高距離分解能化. 日本音響学会誌, 2011, 67.4: 152-154.
\bibitem{安田靖彦} 安田靖彦. 自己同期符号について. 1967.
\bibitem{mseq} M系列とその応用 / 柏木濶 著 昭晃堂 1996 (センシング/認識シリーズ , 第8巻)
\bibitem{眞田幸俊} 眞田幸俊. RAKE 受信方式の原理と発展. 電子情報通信学会 基礎・境界ソサイエティ Fundamentals Review, 2011, 5.1: 20-27.
\bibitem{羽渕裕真} 羽渕裕真. M 系列を基に構成される系列とその通信への応用. 電子情報通信学会 基礎・境界ソサイエティ Fundamentals Review, 2009, 3.1: 32-42.
\bibitem{柏木濶} 柏木濶. M 系列再発見. 計測と制御, 1981, 20.2: 236-245.
\bibitem{長棟章生} 長棟章生, 手塚浩一. M 系列信号パルス圧縮方式地中探査レーダ. 計測自動制御学会論文集, 1994, 30.10: 1151-1157.
\bibitem{土井恭二} 土井恭二, 木村憲明, 弓井孝佳. デジタル符号方式レーダの開発. 三井造船技報, 2011, 202: 20-25.
\bibitem{大槻茂雄} 大槻茂雄, 奥島基良. M 系列変調超音波ドプラ速度計. 日本音響学会誌, 1973, 29.6: 347-355.
\bibitem{青島伸治} 青島伸治, 五十嵐寿一. M-系列の相関を用いた音響測定. 日本音響学会誌, 1968, 24.4: 197-206.
\bibitem{樊春明} 樊春明, 安田明生. 測距信号としての M 系列の検出性能. 電子情報通信学会論文誌 B, 1997, 80.4: 361-367.
\bibitem{musen} 無線通信とディジタル変復調技術 : 変復調の基礎/スペクトル拡散通信/CDMA,OFDM,UWB / 石井聡 著 CQ出版 2005 (RFデザイン・シリーズ)
\bibitem{sensing} センシングのための情報と数理 / 出口光一郎, 本多敏 共著 コロナ社 2008 (計測・制御テクノロジーシリーズ , 2)
\bibitem{keisokusensa} 計測・センサにおけるディジタル信号処理 / 近藤倫正 [ほか]共著 昭晃堂 1993 (ディジタル信号処理シリーズ , 第12巻)
\bibitem{gpsp} GPSのための実用プログラミング / 坂井丈泰 著 東京電機大学出版局 2007
\bibitem{cdt} C言語によるディジタル無線通信技術 / 神谷幸宏 著 コロナ社 2010
\bibitem{cheun97} Kyungwhoon Cheun. Performance of Direct-Sequence Spread-Spectrum RAKE Receivers with Random Spreading Sequences. IEEE Transactions on communications, vol.45, no.9, pp.1130--1143, 1997.
\end{comment}

% 音声通信
\begin{comment}
% \bibitem{中西俊之} 中西俊之. 6500m 潜水調査船システムの超音波技術. 日本音響学会誌, 1996, 52.2: 131-136.
% \bibitem{FRIGG} FRIGG, Roman, GROSS, Thomas R., MANGOLD, Stefan. Multi-channel acoustic data transmission to ad-hoc mobile phone arrays. In: ACM SIGGRAPH 2013 Mobile. ACM, 2013. p. 20.
% \bibitem{松廣紀夫} 松廣紀夫. 超音波による魚群探知技術 (< 小特集> 音響センシングが拡げる測定技術の世界: 超低周波から超音波領域まで). 日本音響学会誌, 2005, 61.11: 665-670.
% \bibitem{越智寛} 越智寛. 海洋観測システムでの 「音」 の利用: 自律型無人探査機 (AUV) の水中通信技術. 日本音響学会誌, 2004, 60.12: 735-740.
\end{comment}


% 通知
\begin{comment}
% \bibitem{umezu11}   梅津直貴, 井ノ上寛人, 堀内恒, 佐藤美恵, 小黒久史, 春日正男.空間把握性に注目した音響案内システムの開発に関する研究.映像情報メディア学会技術報告 vol.35, no.39, pp.41--44,2011.
\end{comment}

% 聴覚
\begin{comment}
% \bibitem{末廣大地} 末廣大地, 翁長博, 池田哲朗. 音楽ホールにおける音に包まれた感じに対応する物理指標の検討. 日本建築学会環境系論文集, 2006, 599: 1-7.
% \bibitem{藤岡繁夫} PA音響システム / 藤岡繁夫 編 工学図書 1996
% \bibitem{羽入敏樹} 羽入敏樹. 室内音響指標値. 日本音響学会誌, 2004, 60.2: 72-77.
% \bibitem{soundsystem} サウンドシステムデザイン / The Bose professional sound group 著,永田穂 訳 オーム社 1991
% \bibitem{onzou}     平原達也, 蘆原郁, 小澤賢司, 宮坂榮一, 音と人間, 日本音響学会編, コロナ社, 2013.
% \bibitem{damasuke} スピーカ・システム / 山本武夫 編著 ラジオ技術社 1977 (ラジオ技術選書 , 114,115)
% \bibitem{raylei1877} Lord Rayleigh, Acoustical Observations I, Philosophical Magazine Series 5, vol. 3, Issue 20, pp.456--464, 1877. %http://www.tandfonline.com/doi/abs/10.1080/14786447708639268?journalCode=tphm16
% \bibitem{raylei1907} Lord Rayleigh, On our perception of sound direction, Philosophical Magazine Series 6, vol. 13, Issue 74, pp.214--232, 1907. %http://www.tandfonline.com/doi/abs/10.1080/14786440709463595#abstract %[Lord Rayleigh: On our perception of sound direction, Phil.Mag, 13,6th series,pp.214-232(1907)]
% \bibitem{播摩敏雄} 播摩敏雄, 安倍幸治, 高根昭一, 曽根敏夫, 音像定位における先行音効果とエコー知覚の限界に関する考察, HIP 104(526), pp.13--18, 2004. %http://ci.nii.ac.jp/els/110003272588.pdf?id=ART0003775596&type=pdf&lang=jp&host=cinii&order_no=&ppv_type=0&lang_sw=&no=1451791039&cp=
\end{comment}




% パニング

% 音声すかし
% \bibitem{茂出木敏雄07} 茂出木敏雄, 千葉誠. 音脈分凝を活用した音楽電子透かし技術" ゲンコーダ Mark" の開発 (アプリケーション). 情報処理学会研究報告. EC, エンタテインメントコンピューティング, 2007, 2007.37: 89-96.

% ネットワーク



% システム

%http://sonove.angry.jp/about_localization.html
%http://acousticslab.org/psychoacoustics/PMFiles/Module07b.htm

\end{thebibliography}
%\end{bibliography}{}
