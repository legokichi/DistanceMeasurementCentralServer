本システムは多端末の時刻同期と測距を音声パルスで実現し,端末のスピーカを用いた音像定位を実現する.
まず,スマートデバイスのスピーカを用いた音像定位手法として,仮想音源と各スピーカとの距離から距離減衰を算出し振幅による音像定位をするDBAP法を用いた\cite{dbap}.
これを用いてデバイス間で指定音源位置から音を鳴らすためには,遅延のないよう端末間同期,仮想音源と各端末間の距離推定,
および各スピーカ間の音圧レベル制御が必要である.


スマートデバイスやセンサネットワークなどでの時刻同期の手法として,音声パルスの往復による時刻同期手法が提案されている\cite{tpsn}.
二つの端末間で音声パルスを交互に鳴らし,互いの端末でのパルスの到来時刻から標準大気圧下での音速における伝搬時間の差を利用し相対距離を求める.
この手法を用い,各端末が一回ずつ音声パルスを鳴らし互いのパルスの到来時刻とその差を計測し,時刻同期と相対距離計測を行った.


上記の手法で判明した各端末の相対距離から,非計量多次元尺度構成法を用いて空間上での相対的な分布を推定することで,
距離測定誤差があったり距離が離れすぎていて音声パルス検出に失敗する場合にも対応しようと考えた.
非計量多次元尺度法では,計測相対距離と推定位置の相対距離の差を取る誤差関数を最少化する非線形最適化問題として定式化し,最急降下法で解く.


高精度な測距には,パルスの到来時刻の正確な検出が必要である.その一方室内環境は環境雑音や壁や天井などに反射によるマルチパスによる問題が生じる.
そこで,環境雑音にも強い擬似雑音系列を用いた直接スペクトル拡散方式によるパルス圧縮を音声パルスに用いた.
そしてマルチパス環境下でのパルス位置推定手法として,Rake 受信器にも使われる伝送路測定用信号と計測信号の二つを用意する手法を応用し,
それらの信号を互いに短時間相互相関にかけることで,厳密なパルス到来時刻の同定を試みた.


音源定位においては,異種端末間でのハードウェア・ソフトウェアに起因する差異の影響を考慮する必要がある.
例えば,スピーカ出力やシステムクロックの異なる端末間では, DBAP 法の適用上問題や同期後の時間経過による同期ズレ発生の問題がある.
このような端末間差異の校正手法としてスピーカ出力に関しては同期信号の距離減衰から各端末の相対音圧を推定する手法を,
同期ズレに対しては一定時間後に再同期することでクロック差を検出する手法を検討した.


これらの制御システムとして,通信を中継する中央サーバと計算サーバを導入したスター型ネットワーク構成と, P2P リングネットワークによる構成の二つを提案した.
