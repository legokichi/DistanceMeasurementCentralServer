\chapter{考察}
%研究全体の考察
\begin{comment}
既知の問題としては,Nexus7が録音中に頻繁にバッファを取りこぼすことが挙げられる.

%![](img/nexus7_is_bad.png)

この現象によりパルスが到来した時刻が分からないので,上記手法が使えなかった.
原因はハードウェアにあるのかOSにあるのかブラウザの実装にあるのか不明であり,現在調査中である.
MacBookAirとMacBookPro間では同期に成功している.



他には,TDMAのみではN回の排他的パルス送信が必要で同期に時間がかかるという問題がある.
これにはスペクトル拡散を使っているのでCDMA(Code Division Multiple Access)化できる余地があるが,
今までの実験で,Nexus7ではGold符号でBPSKした信号が重なったときに,
分離検出できないという問題があったため,開発は滞っている.

\end{comment}

まず,本研究で狙いとするスマートデバイスを用いた音像定位手法として,
端末間同期において提案手法による高精度な同期・測距が実現した.
また,その同期を果たした複数端末による音像定位は,同一の音源として聞かせることができた.

受聴者からみた三角形のみかけの幅が大きい場合にASWが増加したのは,
ASWの要因である二つのノード間で両耳間相関度(ICC: interaural cross-correlation)が
%不可干渉
%インコヒーレントだからだと考えられる<<要出典>>.
小さかった\cite{morimoto95},もしくは,
もう一つの要因として初期側方エネルギー率が大きい,
つまり直接音の入射角度に対して垂直な成分の音圧レベルと直接音の音圧レベルの比が大きかった\cite{barron81}
と考えられる.
%という要因<<要出典>>も考えられる.
また,
三角形の内側の受聴者がLEVを体験したのは,
LEVの要因である前後エネルギー比\cite{suehiro06}
が小さくなったからと考えられる.


\section{barker coded chirpについて}
提案手法で使われるパルス圧縮がbarker coded chirpからM系列符号による直接拡散方式に変わった理由を述べる.

本来barker codeはそのままサイン波に適用し直接拡散して使うものであるが,
系列長が最大で13までしかないため,パルス圧縮に上限があった.
一方でchirp信号は引き伸ばせば引き伸ばすほどパルス圧縮されるが,サイドローブが大きくなるという問題を抱えていた.
そのため,この二つの手法を組み合わせて,短いup-chirpの繰り返しをbarker codeで拡散することで,
サイドローブを抑えながら圧縮する手法を試みた.
しかしながらこの手法は高周波成分において非線型歪みの影響を受けやすく,複数のピークが現れてしまうという問題を抱えていた.
%要図

一方で,M系列符号には系列長に制限がないため,そのような手法を必要とせず,直接拡散方式が利用できる.
以前の実装は私がbarker codeによるパルス圧縮を知った時点で,まだM系列符号を知らなかった故のものであり,
M系列符号が使える現在,そのようなハイブリット手法を使う理由はない.
