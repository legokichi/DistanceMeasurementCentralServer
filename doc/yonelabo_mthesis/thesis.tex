\documentclass[11pt]{jreport}
%\documentclass[11pt,twoside]{jreport} %印刷に出すときは両面見開きにするかな
\usepackage{fancyhdr}
\usepackage{graphicx}
\usepackage{epsf}
\usepackage{url}

\let\origtitle\title
\let\origauthor\author
\newcommand{\thetitle}{a}
\newcommand{\theauthor}{a}

\renewcommand{\bibname}{参考文献}
\makeatletter
\renewcommand{\chapter}{%
  \if@openright\cleardoublepage\else\clearpage\fi
  \thispagestyle{fancy}
  \global\@topnum\z@
  \@afterindenttrue
  \secdef\@chapter\@schapter}
\makeatother

 \pagestyle{fancy}
 \fancyhead{}
  \rhead{\footnotesize\thepage}
  \lhead{\footnotesize\leftmark}
  \chead{\footnotesize\rightmark}
  \cfoot{}
  \renewcommand{\chaptermark}[1]{\markboth{第\ \normalfont\thechapter\ 章~#1}{}}
  \renewcommand{\sectionmark}[1]{\markright{\thesection #1}{}}
\renewcommand{\headrulewidth}{0.4pt}
\renewcommand{\footrulewidth}{0.4pt}
\renewcommand{\title}[1]{\lfoot{\footnotesize{#1}\origtitle{#1}}\renewcommand{\thetitle}{#1}}
\renewcommand{\author}[1]{\rfoot{\footnotesize{#1}\origauthor{#1}}\renewcommand{\theauthor}{#1}}
%\lhead[{\footnotesize\hspace{1em}\thepage\hspace{2em}\rightmark}]{} 
%\rhead[]{{\footnotesize\hspace{1em}\leftmark\hspace{2em}\thepage}} %見開き用
\newcommand{\athr}[1]{\author{#1}}



%ここから
\title{納豆の糸の数え方についての自動糸追跡手法に関する研究}
\author{米澤朋子}
\date{\today}
\let\origtitle\title
\let\origauthor\author

\begin{document}
\thispagestyle{empty}
\begin{center}
\huge{平成 26 年度}\\
\Huge{修 士 論 文 題 目}\\
\vspace{1.5cm}\huge{\thetitle}\\ \vspace{3.0cm}
\LARGE{関西大学大学院 総合情報学研究科}\\
\LARGE{博士前期課程}\\
\LARGE{知識情報学専攻}\\
\vspace{0.8cm}\Large{インタラクションの認知・メディア・文化}\\
\vspace{0.8cm}\LARGE{\theauthor}\\ \vspace{1.0cm}
\today\vspace{1.5cm}
\begin{itemize}
\LARGE{主査\hspace{2cm} 米澤朋子 准教授}\\
\LARGE{副査\hspace{2cm} 加藤隆 教授}\\
\LARGE{副査\hspace{2cm} 喜多千草 教授}\\
\end{itemize}
\end{center}


%もくじ
\tableofcontents
\listoftables
\listoffigures


%内容
\chapter*{概要}
\addcontentsline{toc}{chapter}{概要}

本研究では,複数のスマートデバイスを用いてスピーカアレイを構築し,音源配
置による音声情報提供システムを提案する.
スピーカアレイには正確な位置情報と端末間同期が必須である.
本稿では,直接スペクトル拡散方式によるパルス圧縮を用いて
各端末のスピーカとマイクロフォンを用いて同期と測距を行い,
最急降下法を用いて相対位置推定をした.
また,振幅パニングで相互の位置に応じた音像定位することで,
仮想音源配置と音声情報提供を
実空間内の複数端末によるスピーカアレイにより実現した.

\chapter*{発表文献リスト}
\addcontentsline{toc}{chapter}{発表文献リスト}

\begin{itemize}

  \item[] ジャーナル論文
  \begin{enumerate}
    \item 中祐介, 伊納洋佑, 吉田直人, 米澤朋子, 身体動作・環境音のオノマトペを含むテキストコミュニケーション手法の検討, HI学会論文誌, vol.17, no.2, pp 97--106, 2015.
  \end{enumerate}

\begin{comment}
  \item[] 国際会議論文
  \begin{enumerate}
    \item hoge
  \end{enumerate}
\end{comment}


  \item[] 国内会議論文
  \begin{enumerate}
    \item 伊納洋佑, 石川佑樹, 中祐介, 米澤朋子, 複数の携帯端末の同期・位置推定手法による閉鎖空間の音響環境構築, 電子情報通信学会応用音響研究会研究会, vol. 115, no. 424, EA2015-58, pp. 19-26,	2016.
    \item 石野力, 伊納洋祐, 米澤朋子, 空間指向性を含む繰り返し音楽の制御と演奏効果の検証, 情報処理学会EC-MUS合同研究会, Vol.2015-MUS-106, No.18, pp.1--6, 2015.
    \item 伊納洋祐, 吉田侑矢, 米澤朋子, 複数端末の音響的位置推定と同期による空間音響環境構築システムの提案, 日本音響学会秋季研究発表会2014, pp.1439--1440, 2014.
    \item 岡本直也, 伊納洋祐, 米澤朋子, ディジタル画像への温感付与による非実体の体感システムの提案, 第109回ヒューマンインタフェース学会研究会, SIG-ACI, pp.33--36, 2014.
    \item 塩尻実里,中谷友香梨,吉田侑矢,伊納洋祐,米澤朋子, 動詞の身体性に基づくアニメーション生成手法を適用したピクトグラム表現手法の検討, HIS 2014, 1528D, pp.313--318, 2014.
    \item 伊納洋祐, 吉田直人, 中谷友香梨, 吉田侑矢, 米澤朋子,	複数の携帯端末による教室空間の空間音響環境構築手法の検討,	電子情報通信学会MVE研究会,	MVE2013-38, pp.41--44,	2014.
    \item 河口拓貴, 林亜里紗, 伊納洋祐, 吉田直人, 米澤朋子, 上体の重心移動を伴う身体動作による音楽演奏時のリズム生成手法の提案, 電子情報通信学会MVE研究会, MVE2013-38, pp.49--52, 2014.
    \item 石野力, 伊納洋祐, 中谷友香梨, 吉田直人, 米澤朋子, 複数パラメトリックスピーカを用いた一対多コミュニケーション手法の提案, 電子情報通信学会MVE研究会, MVE2013-38, pp.53--58, 2014.
    \item 加藤貴志, 吉田侑矢, 伊納洋介, 米澤朋子, 話者と聴衆の対面式書き込みプレゼンテーション手法に関する検討, 第101回ヒューマンインターフェイス学会研究会, pp.29--30, 2013.
  \end{enumerate}

  \item[] その他
  \begin{enumerate}
    \item 国際バーチャルリアリティ学生コンテスト 「茶禅空」審査員特別賞受賞,2013.
  \end{enumerate}

\end{itemize}

\chapter{序論}


\section{緒論}
緒論,,どのくらいかくのかな
\clearpage
2ページとか??4ページとか??


\begin{figure}[tb]\centering
\epsfxsize=10cm\epsffile{eps/nattorice.eps}
\caption{元気の源,納豆ごはん}\label{fig:nattorice}
\end{figure}

ところで納豆ごはん(図\ref{fig:nattorice})は大好物ですけど,
ひき割り納豆を使ったアレンジ料理もたまりません.
鎌倉パスタの納豆オクラパスタとか,もう涎がガンガン出ます.だからこうなるんですね.

\section{研究の目的}
ここはかなり大事
\clearpage
わかりやすく,段落を丁寧に区切りましょう

\section{本論文の構成}

\chapter{関連研究}
%または
%\chapter{菌類}

複数のスマートデバイスで複数のユーザにサービスを提供するシステムとして,
位置情報を利用した携帯端末への音声情報配信がある.
このようなサービスは,
ユーザ情報を取得し共有するための
個人を対象としたサービスであり,集団を対象とはせず,
端末間の通信で実空間に刺激を形成するものでもない.

- 河越嵩介,神場知成,田中二郎.位置情報を利用した携帯端末への音声情報配信,
情報処理学会第76回全国大会,4ZA-2,2014.

\section{マルチチャンネルスピーカ}

これまでの多チャンネルスピーカによる音場再現手法としては,
波面合成法(WFS: wave field synthesis),
高次アンビソニックス法(HOA: higher order Ambisonics),
境界音場制御法
などが知られている.
また,
パラメトリックスピーカを用いて特定の場所に音像を定位する手法がある.
これらの手法はどれも特殊な機器と特別な設定が必要であり,
公共空間への導入が困難である.

- Berkhout, Augustinus J., Diemer de Vries, and Peter Vogel. "Acoustic control by wave field synthesis." The Journal of the Acoustical Society of America 93.5 (1993): 2764-2778.
- Daniel, Jérôme. "Spatial sound encoding including near field effect: Introducing distance coding filters and a viable, new ambisonic format." Audio Engineering Society Conference: 23rd International Conference: Signal Processing in Audio Recording and Reproduction. Audio Engineering Society, 2003.
- Ise, Shiro. "A principle of sound field control based on the Kirchhoff-Helmholtz integral equation and the theory of inverse systems." Acta Acustica united with Acustica 85.1 (1999): 78-87.
- 青木茂明,清水一博,伊藤昂輝.パラメトリックスピーカを用いた再生時の音像定位.信学技報 EA研究会,vol.114,no.423, pp.33--38, 2015.

\section{アドホックマイクロホンアレイ}

これまで,端末間同期手法に関して,
音の発信を利用したスマートフォンアレイの機器位置推定や
音の発信を利用したキャリブレーションに基づくアドホックマイクロホンアレイによる音源位置推定がある.
マイクロホンアレイは複数スマートデバイスのマイクロホンで取得した多チャネル信号を処理し,音源位置推定,音源分離などを行う.
これはスマートデバイスでアレイ処理をする点においては似ているが,
本研究ではスピーカアレイを構築するために相対位置推定や同期を行うためのマイクロフォンの利用という点で異なる.

- 柴田一暁, 小野順貴, 亀岡弘和. 音の発信を利用したスマートフォンアレイの機器位置推定. 音講論 (秋), pp.591--592, 2013.
- 柴田一暁, 小野順貴, 亀岡弘和. 音の発信を利用したキャリブレーションに基づくアドホックマイクロホンアレイによる音源定位. 音講論 (春), pp. 707--710,2014.


\section{パルス圧縮手法}
また,先の手法は音による位置測定における測距パルスに
時間引き伸ばしパルス(Time Stretched Pulse:TSP) を使用している.
これは音響測定におけるインパルス応答を測定するための信号であり,
継続時間の長いTSPは相関結果のサイドローブが大きくなるため距離を測定するための信号としては不適当であるという問題がある.

- N. Aoshima, Computer-generated pulse signal applied for sound measurement, J. Acoust. Soc. Am., vol.69, no.5,1484--1488, 1981.


\section{本研究のスタンス}
以上をまとめると,
スマートデバイスを用いた情報提供システムには,個人向けの研究が目立つ.
また,複数端末を用いてマイクロホンアレイを構築する研究は存在するが,複数端末を用いてのスピーカアレイを構築する手法は比較的未開拓分野と言える.
そして,複数のスマートデバイスを用いて実空間内の複数の人間に働きかける,という本研究のシステムは,
今後さらに生活空間にスマートデバイスが普及することを考えると,
パラコミュニケーションを実現する手段としても重要であると言える.


%\section{菌類とは}
%\section{細菌類とは}
%\section{納豆菌とは}
%\section{納豆菌の既知の効能}
%\section{本研究のスタンス}

%\chapter{納豆菌の菌糸直径計測および菌糸本数推定手法}
%\section{納豆菌の直径計測手法}
%\section{納豆菌の菌糸本数推定手法}

\chapter{提案システム}

提案システムでは,教室などの閉鎖空間において,
個人が所有する複数のスマートデバイスの音声出力をネットワークを介し同期させ制御することで,スピーカアレイを構築する(図\ref{fig:shikumi2}).
空間内に配置した仮想音源の位置に基づき,その音源位置を囲む最寄りの3つのスマートデバイス(ノード)を設定し振幅パニングすることで,現実世界における想定位置で音源を鳴らして定位する.
このようなスピーカアレイを構築するにあたり,実空間に分布する複数のスマートデバイスの相対位置を推定するとともに,端末間での時刻同期が必要不可欠である.

\begin{figure}[p]\centering
  \hspace{-2mm}\includegraphics[clip,width=1.1\hsize]{img/shikumi3.png}
  \caption{提案システムの概念}\label{fig:shikumi2}
\end{figure}

この章では,まず,提案システムで用いた
相対位置に基づく音像定位手法について触れる.
次に,その手法を実現するための
端末間の音声パルスの到達時間差による
時刻同期手法および相対距離計測手法,
そして相対位置推定手法を述べる.
さらに,パルス圧縮を用いた信号検出手法を示し,
最後に,スピーカアレイ全体の制御手法について解説する.



\section{DBAP法を用いた音像定位}

はじめに,複数のスマートデバイスを使ってどのように音像定位するのかについて述べる.
平面配置のスピーカアレイを用いて音像定位をする手法として,
Distance-based amplitude panning (DBAP) 法\cite{dbap}がある.

DBAP法は,
任意の数のスピーカの位置が既知であり,
端末間のスピーカの出力特性が等しいときに,
仮想音源と各スピーカとの距離から距離減衰を計算することで,
各スピーカの振幅を制御して音像を合成する手法である.

図\ref{fig:DBAP}のように,位置 $(x_s,y_s)$ にある仮想音源 $VS$ から,
位置 $(x_i,y_i)$ にある端末 $i$ への距離 $d_i$ を次のように定義する.

\begin{figure}[p]\centering
  \hspace{-2mm}\includegraphics[clip,width=1.1\hsize]{img/DBAP.png}
  \caption{DBAP法の図解}\label{fig:DBAP}
\end{figure}

$$
d_i = \sqrt{(x_i - x_s)^2 + (y_i - y_s)^2} \qquad (\mathrm{for}\ 1 \leq i \leq N)
$$

DBAP法では,仮想音源の位置に関係なく,各スピーカからの音の強さの合計を

$$
I = \sum_{i=1}^N v_i^2 = 1
$$

として正規化している.
$i$ 番目のスピーカの相対的な振幅は距離に反比例するので

$$
v_i = \frac{k}{d_i^a}
$$

と定義できる.
$k$ はすべてのスピーカと仮想音源の位置に依存した係数で,

$$
k = \frac{1}{\sqrt{\sum_{i=1}^N \frac{1}{d_i^{2a}}}}
$$

である.
係数 $a$ は振幅の距離減衰係数で

$$
a = \frac{R}{20 \log_{10}2} \\
$$

と定義する.
$R$ はロールオフ係数で,受聴者と音源の距離に基づく減衰の量である.
$R=6\ [\mathrm{dB}]$ の場合は,
自由空間における距離減衰の逆二乗則に基づき,音の強さのレベルが音源からの距離が2倍になるごとに6dBずつ減少することを意味する.
また,半自由空間では $R=3\sim5\ [\mathrm{dB}]$ 程度となる.

以上のとおり,
複数のスマートデバイスが
同期的に制御でき,
端末の位置が判明しており,
端末間のスピーカの出力特性が均一であれば,
この手法を用いて音像定位ができることが分かった.



\section{時刻同期と測距}

基準となる時刻が違う二つの時間軸を持つ端末間において同期するには,
互いに音声パルスを出せばよい(図\ref{fig:beeptobeep}).

\begin{figure}[tb]\centering
  \hspace{-2mm}\includegraphics[clip,width=1.1\hsize]{img/beeptobeep.png}
  \caption{beeptobeep}\label{fig:beeptobeep}
\end{figure}

この手法はTPSN(time-sync Protocol for sensor network)\cite{tpsn}
などで提案されている.
原理を説明する.

端末 $A$ が自身の時刻 $t_0$ に音声パルスを発生すると,
そのパルスは音速で空間に広がり,
端末 $B$ 内の時刻 $t_1$ に受信される.
さらに,端末 $B$ からも端末 $B$ 内の時刻 $t_2$ に音声パルスを発すると,
このパルスも音速で空間に広がり,
端末 $A$ 内の時刻 $t_3$ に受信される.
ここで,図の通り,
端末 $A$ 内のパルス時間間隔 $t_3-t_0$ と
端末 $B$ 内のパルス時間間隔 $t_2-t_1$ には差が生じる(図\ref{fig:clocksynchronization}).

\begin{figure}[tb]\centering
  \hspace{-2mm}\includegraphics[clip,width=1.1\hsize]{img/clock_synchronization.png}
  \caption{clocksynchronization}\label{fig:clocksynchronization}
\end{figure}

パルスの往復で伝播にかかった時間は共に等しいと仮定すると,

$$
t_0' = t_1 - \frac{(t_3 - t_0) - (t_2 - t_1)}{2} \\
$$

となり,端末 $B$ 内時刻で端末 $A$ のパルスが発せられた時刻を推定することができる.
以上が時刻同期の原理である.

音速を $c$ とすれば,副次的に端末間の距離も求まる.

$$
d_{AB} = \frac{(t_3 - t_0) - (t_2 - t_1)}{2c}
$$

端末AB間で何らかの処理を同期的に実行したい場合は
図のようにして実行すべき時間を求められる(図\ref{fig:flowchart3}).

\begin{figure}[tb]\centering
  \hspace{-2mm}\includegraphics[clip,width=1.1\hsize]{img/flowchart3.png}
  \caption{同期実行}\label{fig:flowchart3}
\end{figure}

基準となる端末Aのパルスの受信時間およびその信号の伝達時間と,
それを受信してからの経過時間 $S$ をもとに,同期的に処理を実行できる.

こうして,端末間の時刻同期と距離測定ができた.



\section{相対位置推定}


次に,複数のスマートデバイスの空間分布をどう推定するかについて述べる.
次のように定式化し誤差関数を最小化する最適化問題を考える.

$$
\varepsilon(\hat{x_1}, \dots, \hat{x_N}) = \sum_{i=1}^N \sum_{j\in M(i)} \left( \| \hat{ x_i } - \hat{ x_j } \| - d_{ij} \right)^2 \\
%\DeclareMathOperator*{\argmin}{arg\,min}
(\hat{x_1} \dots \hat{x_N}) = \mathrm{argmin} \varepsilon(\hat{x_1} \dots \hat{x_N})
$$

ここで,
$N \in \mathbb{N}$ は端末の数,
$M(i) \subset \{1,\dots,N\}$ は端末 $i$ と相対距離が計測できた端末の集合,
$d_{ij} \in \mathbb{R}$ は実際に計測された端末の距離とし,
$\hat{ x_i } \in \mathbb{R}^2$ n番目の端末の位置推定値で,初期値は乱数を置く.

最急降下法を使って反復的に解く.更新式は次のようになる.

$$\begin{aligned}
\hat{x_i} (n + 1) & = \left. \hat{x_i} (n) - a \frac{\partial \varepsilon}{\partial \hat{x_i}} \right|_{\hat{x} = \hat{x}(n)} \\
\frac{\partial \varepsilon}{\partial \hat{x_i}}
&= \sum_{j\in M(i)} \frac{\partial \left( \|\hat{ x_i } - \hat{ x_j }\| - d_{ij} \right)^2}{\partial \hat{x_i}} \notag\\
&= 2 \sum_{j\in M(i)} \left( \| \hat{x_i} - \hat{x_j} \| - d_{ij} \right) \frac{\partial \| \hat{x_i} - \hat{x_j} \|}{\partial \hat{x_i}} \notag\\
&= 2 \sum_{j\in M(i)} \left( 1 - \frac{d_{ij}}{\| \hat{x_i} - \hat{x_j} \|} \right) \left( \hat{x_i} - \hat{x_j} \right).
\end{aligned}$$

$n$ は反復回数, $a$ は更新式のステップ幅である.
図\ref{fig:relpo_s}にすべての端末間で距離が取得できたとしたシミュレーション結果を示す.


\begin{figure}[p]\centering
  \hspace{-2mm}\includegraphics[clip,width=1.1\hsize]{img/positiondetection-new.png}
  \caption{最急降下法によって相対位置を解いたシミュレーション結果}\label{fig:relpo_s}
\end{figure}

端末 $i$ が測距できた他の端末の集合 $M(i)$ は後述する信号検出により決まる.


\section{信号検出}

\subsection{必要な同期制度のための信号検出の要件}

精密な測距・時刻同期のためには精密な信号検出が必要である.
まず許容される信号検出の誤差について考える,

スマートデバイスのサンプリング周波数を44100Hz
として
1サンプルあたりの時間解像度は約 1/44100 = 22.6μs
であるので,
1サンプルあたりの距離解像度は 22.6μs*340m/s=7.7mm(音速340m/sと仮定)
である.

人間の聴覚特性として,第一波面の法則という現象が知られている\cite{Haas}.
これは二つの音源からの音声が互いに50ms以上ずれると別の音源として知覚されるというものである.

この場合許容される誤差は
$\pm$ 50msの誤差におよそ $\pm$ 2205サンプル以内とかなり緩いものになる.
しかしこれでは距離誤差が $\pm$ 3.4m となり,教室空間での位置推定のパラメータとして使うにはとても大きなものになってしまう.
そのため,許容される誤差は音像定位ではなく距離測定の手法によって定められるといえる.
室内空間であれば,
$\pm$ 50cmの誤差つまり $\pm$ 64サンプル以内でパルスを同定できればよいとする.


\subsection{本研究で試したパルス圧縮の種類}

本研究で試行したパルス圧縮を表 \ref{tab:pulsecomps} にまとめる.

\begin{table}[p]\centering
  \caption{本研究で試したパルス圧縮の種類}
  \label{tab:pulsecomps}
  \begin{tabular}{ll|ccc}
    \hline
    パルス & 利点 & 欠点 \\
    \hline
    \parbox{10zw}{固定周波数パルス}
      & \parbox{10zw}{可聴域外のパルスに設定可能}
      & \parbox{10zw}{ノイズに弱い,ピーク発見が困難, \\ 可聴域外音声の発生・受信が安価な端末では困難}
      \\
    \hline
    \parbox{10zw}{全帯域チャープ信号}
      & \parbox{10zw}{パルス圧縮によりノイズに強い}
      & \parbox{10zw}{信号強度確保のために \\ 持続時間を長くするとピークが鈍くなる}
      \\
    \hline
    \parbox{10zw}{バーカー符号を用いた \\ 直接スペクトル拡散}
      & \parbox{10zw}{鋭いピークを持たせられる}
      & \parbox{10zw}{系列長が13までしかないため \\ SN比に限界がある}
      \\
    \hline
    \parbox{10zw}{バーカー符号化\\ チャープ信号}
      & \parbox{10zw}{短いチャープをバーカー符号化することで \\ チャープのピークを鋭く保ったまま13倍にできる}
      & \parbox{10zw}{高周波数帯の非線形歪みの影響で \\ 複数のピークが現れる}
      \\
    \hline
    \parbox{10zw}{M系列を用いた\\ 直接スペクトル拡散}
      & \parbox{10zw}{パルス持続時間を伸ばしてSN比を高めてもピークが鋭い}
      & \parbox{10zw}{持続時間を長くしすぎると非線形歪みの影響で \\ 複数のピークが現れる}
      \\
    \hline
  \end{tabular}
\end{table}

まず可聴域外に近い20000Hz前後の固定周波数パルスを用いた同期信号を試みた.
しかし,安価なスマートデバイスのスピーカとアンプとマイクロホンでは送受信ができない,
可聴域音にしても信号の検出が困難などの理由で不採用となった.

次に,チャープ信号を用いたパルス圧縮手法を試みた.
これにより信号強度を高める事ができた.
だが,信号のサイドローブが大きく鋭いピークを持たせられなかった.

チャープ信号に代わるもとして,バーカー符号を用いた直接スペクトル拡散を考えたが,
バーカー符号は系列長13までしかないので,そのまま直接拡散してもパルス持続時間が短くSN比を向上できない.
そこで,直接拡散において符号化する1波長を通常の正弦波ではなくひとつのチャープ信号にするパルス圧縮信号を試みた.
しかし,これも高周波数帯域の非線形歪みの影響を受け,複数のピークが現れた.

最後に,M系列符号を用いた直接スペクトル拡散を試みた.
これにより,ピークを鋭く保ったまま任意の持続時間のパルスを作ることができるようになった.
しかし,持続時間をむやみに長くすると,非線形歪みの影響を受けるようになった.

これらを順に説明する.

\subsection{固定周波数パルス}

本研究ではまず固定周波数パルスを試みた.
周波数を可聴域外に近い20000Hz前後にすることで,
利用者の耳には聞こえにくい同期信号を利用できると考えたからである.
これは実際MacBookProなどの高品質な端末では送受信可能だったが,
Nexus7(2013)などの比較的安価な端末のスピーカやマイクでは送受信できなかった.
また,この信号は可聴域としてもピークが鈍くで同期信号には適さなかった.

\subsection{チャープ信号}

そこで,チャープ信号を用いたパルス圧縮を試みた.
チャープ信号は時間に対して周波数を変化させた信号で,
同じ長さの固定周波数パルスに対してピークを鋭くすることができる利点がある.
しかし.持続時間を長くするとピークが鈍くなってしまう.
これによりマルチパスの影響により現れる複数のピークが重なってしまい,最初に届いた波のピークを特定することを困難とさせていた.
このため,この同期・測距信号には適さなかった.

\subsection{バーカー符号化チャープ信号}

そこでさらに,バーカー符号とチャープ信号のふたつのパルス圧縮方式を組み合わせる独自のパルス圧縮を考案した\cite{self_ac}.
バーカー符号とチャープ信号の二つのパルス圧縮技術を組み合わせて,チャープ信号をバーカー符号を用いてBPSKで変調したものである.
バーカー符号は直接スペクトル拡散方式で用いられる拡散系列で,
Diracのデルタ関数に近いするどい相関特性を持つ.
しかし,このようなパルスは系列長13までしか存在しないので,そのまま直接拡散しても持続時間を長くできずSN比を向上させられない.
そこで,バーカー符号を用いてBPSKして直接拡散するにあたって,チップ幅を1波長の正弦波ではなく,ひとつのチャープ信号を用いる事で,
チャープ信号の鋭さを保ったまま通常のチャープ信号の13倍のSN比を稼げると考えた
(図\ref{fig:barker_chirp},\ref{fig:barker_coded_chirp},\ref{fig:barker_coded_chirp_err},\ref{fig:corr}).

\begin{figure}[p]\centering
\includegraphics[clip,width=1.0\hsize]{img/barker_chirp.png}
\caption{チャープ信号を搬送波として長さ13のバーカー符号でBPSK変調信号したもの}\label{fig:barker_chirp}
\end{figure}

\begin{figure}[p]\centering
\includegraphics[clip,width=0.75\hsize]{img/barker_coded_chirp.png}
\caption{チャープ信号を搬送波として長さ13のバーカー符号でBPSK変調信号したもののスペクトルグラム(シミュレーション)}\label{fig:barker_coded_chirp}
\end{figure}


\begin{figure}[p]\centering
\includegraphics[clip,width=0.75\hsize]{img/barker_coded_chirp_err.png}
\caption{チャープ信号を搬送波として長さ13のバーカー符号でBPSK変調信号したもののスペクトルグラム
(MacBookAirによる実環境での自身の信号の計測結果)}\label{fig:barker_coded_chirp_err}
\end{figure}



\begin{figure}[p]\centering
\vspace{2mm}
\begin{small}
\includegraphics[clip,width=1.0\hsize]{img/rawdata.png}\\
A端末自身のパルス~ ~ ~ ~
B端末のパルス~ ~ ~ ~ ~
C端末のパルス\\\vspace{0.5mm}
\includegraphics[clip,width=0.8\hsize]{img/spectrogram.png}\hspace{1cm}\\
A端末自身のパルス~ ~ ~ ~ ~ ~ ~ ~ ~
B端末のパルス\\\vspace{0.5mm}
\includegraphics[clip,width=1.0\hsize]{img/corrA.png}\\
A端末のパルスに整合フィルタをかけたもの\\\vspace{0.5mm}
\includegraphics[clip,width=1.0\hsize]{img/corrB.png}\\
B端末のパルスに整合フィルタをかけたもの\\\vspace{0.5mm}
\includegraphics[clip,width=1.0\hsize]{img/corrC.png}\\
C端末のパルスに整合フィルタをかけたもの\\
\vspace{1mm}
\caption{A端末で観測したABC端末のパルス信号}\label{fig:corr}

\end{small}
\vspace{1mm}
\end{figure}

提案手法のパルス圧縮に関して,
以前利用していたこのバーカー符号化チャープ信号によるパルス圧縮ではなく,
M系列符号による直接拡散方式を用いた理由について以下で述べる.

本来barker codeはそのままサイン波に適用し直接拡散して使うものであるが,
系列長が最大で13までしかないため,パルス圧縮に上限があった.
一方でチャープ信号は引き伸ばせば引き伸ばすほどパルス圧縮されるが,サイドローブが大きくなるという問題を抱えていた.
そのため,この二つの手法を組み合わせて,短いup-chirpの繰り返しをbarker codeで拡散することで,
サイドローブを抑えながら圧縮する手法を試みた.
しかし,この手法は高周波成分において非線型歪みの影響を受けやすく,複数のピークが現れてしまうという問題を抱えていた.
%要図

一方で,M系列符号には系列長に制限がないため,そのような手法を必要とせず,直接拡散方式が利用できる.
%以前の実装は私がbarker codeによるパルス圧縮を知った時点で,まだM系列符号を知らなかった故のものであり,
%M系列符号が使える現在,そのようなハイブリット手法を使う理由はない.





\subsection{M系列による直接スペクトル拡散}

本研究ではM系列符号による直接スペクトル拡散方式によるパルス圧縮を用いた.
これはチャープ信号よりも非定常雑音に強い\cite{nonlinear}.
また,信号検出には通常の整合フィルタではなくピークを尖らせることができる,
フェイズオンリー整合フィルタ(phase-only matched filter: POF)\cite{pof, phaseonly2}を利用した(図\ref{fig:pof}).
フェイズオンリー整合フィルタは,整合フィルタに信号の周波数成分のみを利用することでサイドローブを抑えピークをとがらせることができるフィルタである.

\begin{figure}[p]\centering
  \hspace{-2mm}\includegraphics[clip,width=1.1\hsize]{img/POF.png}
  \caption{フェイズオンリー整合フィルタと整合フィルタ}\label{fig:pof}
\end{figure}

$$
\begin{aligned}
\mathrm{POF}[x_a, x_b]
&= \mathcal{F}^{-1}\left[\frac{\mathcal{F}\left[x_a(t)\right]^*}{|\mathcal{F}\left[x_a(t)\right]|}\mathcal{F}\left[x_b(t)\right]\right] \\
&= \mathcal{F}^{-1}\left[\frac{X_a^*(\omega)}{|X_a(\omega)|}X_b(\omega)\right]
\end{aligned}
$$


直接スペクトル拡散方式(direct sequence spread spectrum:DSSS)(図\ref{fig:DS})による測距システムの変復調方式を図\ref{fig:DME2}に示す.
搬送波には4410Hz正弦波を,拡散符号には符号長1023のM系列を用い,変調にはバイナリ位相シフトキーイング(BPSK)を使った.

\begin{figure}[p]\centering
  \hspace{-2mm}\includegraphics[clip,width=1.1\hsize]{img/DME2.png}
  \caption{変復調システム図}\label{fig:DME2}
\end{figure}

\begin{figure}[p]\centering
  \hspace{-2mm}\includegraphics[clip,width=1.1\hsize]{img/DSSS.png}
  \caption{直接スペクトラム拡散による信号変調過程}\label{fig:DSSS}
\end{figure}

この手法により,通常のパルスやチャープ信号,バーカー符号化チャープよりも鋭くSN比の高いピークが得られるようになった.
しかし,まだ問題は残っており,継続時間を長くしすぎると送受信側のシステムクロックのずれにより非線形歪みの影響を受け,ピークが複数現れるようになってしまう.
そこで



\subsection{伝送路推定用の参照信号を用いた信号同定}

復調した受信信号が雑音か有効な信号かを決定する処理を信号同定という.
伝送路における伝達関数 $H(\omega)$ において室内残響の影響
としてマルチパスによる影響を受けてしまう(図\ref{fig:multipath}).

\begin{figure}[p]\centering
  \hspace{-2mm}\includegraphics[clip,width=1.1\hsize]{img/multipath.png}
  \caption{反射波・回折波によるマルチパスの影響}\label{fig:multipath}
\end{figure}


そこで,同期パルスを測距用信号と,伝搬路を測定する参照波の二つに分離した.
参照波を用いた伝送路推定にはCDMA通信でのRake受信器において使われているサウンダ信号や,
OFDM通信などではパイロット信号が推定信号として使われる.
同期パルスから $n$ 秒後に参照信号を送り,
その二つの信号の相関を取ることで,
背景雑音とは別にパルス位置を特定することが可能になる(図\ref{fig:sounder}).

\begin{figure}[p]\centering
  \hspace{-2mm}\includegraphics[clip,width=1.1\hsize]{img/sounder.png}
  \caption{同期信号と参照信号}\label{fig:sounder}
\end{figure}

さらに参照信号と測距信号は互いに異なる同周期のM系列を用いた.
参照信号と測距信号を判別しやすくするためである.

この時刻が $n$ 秒ずれた二つの信号に対して時間窓で区切って相互相関をとることで,
信号が最も相関している区間,つまり信号の位置を特定することができる.
最後に,その区間相関値を閾値処理することで信号の到来を決定する.
今回は最大相関値前方での40\%の相関値を超えたピークを到来時刻としている.

相関関数の計算には Wiener-Khintchine の定理を使い,周波数領域での複素乗算としてFFTを使って計算することで計算量を減らすことができる.
長さの異なる信号の高速フーリエ変換には重畳加算法\cite{overwrap}を使った.


\begin{figure}[p]
  \centering
  \includegraphics[clip,width=1.05\hsize]{img/026.jpeg}
  \caption{変復調回路}\label{fig:henpuku}
\end{figure}


\begin{figure}[p]
  \centering
  \includegraphics[clip,width=1.05\hsize]{img/029.jpeg}
  \caption{信号同定過程}\label{fig:shousai}
\end{figure}




\clearpage


\section{システムクロック校正}

端末のシステムクロックの進みかたはハードウェアごとに微妙に異なる.
そのため,同期してから長時間経つと,次第に端末間で遅延が生じる.
このことは,同一音源を同期的に再生し続けると,次第にずれが聞き取れるようになってくることを意味する.
この遅延を検出する手法について述べる.

端末 A のクロックを $S_A$,
端末 B のクロックを $S_B$ して,
時刻 $t$ 後に
A のサンプル数が $i$ ,
B のサンプル数が $i+d$ だけ異なっているとする(図\ref{fig:phaseshift2}).
このとき,Aを基準とした遅延比率 $S_B/S_A$ を求めたい.

\begin{figure}[tb]\centering
  \hspace{-2mm}\includegraphics[clip,width=1.1\hsize]{img/phase_shift2.png}
  \caption{phaseshift2}\label{fig:phaseshift2}
\end{figure}

図より

$$\begin{aligned}
\frac{i}{S_A} &= \frac{i+d}{S_B} = t \\
\frac{S_B}{S_A} &= \frac{i+d}{i}
\end{aligned}$$

である.
遅延の検出においては,端末間の相対距離に変化がない限り,
端末Aが端末Bへとパルスを発生するだけで良く,
端末BはAへと返答パルスを返す必要はない.
既に同期済みでありAからBへの伝搬時間は算出済みだからである.



\section{音圧校正}
スマートデバイス毎にマイクロホンやスピーカのアンプ出力は異なるため,そのままではDBAP法は使えない.
ここでは,機器ごとの音圧を校正する手法を示す.

LTI(線形時不変)システムを仮定する.
$N$ 台の端末の番号を $i,j \in \{1\dots N\}$ とする.
端末 $i$ から端末 $j$ への信号伝達を考える.
$e$ を端末 $i$ で生成した単位振幅,
$v_i$ を端末 $i$ のスピーカアンプの増幅係数,
$m_j$ を端末 $j$ のマイクロホンアンプの増幅係数,
$d_{ij}=d_{ij}$ を $ij$ 間の測定距離,
$x_{ij}$ を $j$ が観測した $i$ からの信号の振幅とする.
音波の振幅は距離に反比例して減衰することが知られているので,
音声信号の伝達は

\begin{figure}[tb]\centering
  \hspace{-2mm}\includegraphics[clip,width=1.1\hsize]{img/sound_pressure_calibration.png}
  \caption{soundpressurecalibration}\label{fig:soundpressurecalibration}
\end{figure}

$$
e v_i \frac{1}{d_{ij}} m_j = x_{ij}
$$

とモデル化できる(図\ref{fig:soundpressurecalibration}).
このとき,ある端末 $k$ の出力係数 $v_k$ と他の端末 $i$ の出力係数 $v_i$ との比 $v_i/v_k$ を求めたい.

$$
e v_i m_j = x_{ij}d_{ij}
$$

なので

$$\begin{aligned}
\frac{e v_j m_j}{e v_k m_j} &= \frac{x_{ij} d_{ij}}{x_{kj} d_{kj}} \\
\frac{v_j}{v_k} &= \frac{x_{ij} d_{ij}}{x_{kj}d_{kj}} \\
\end{aligned}$$

である.
観測した振幅 $x_{ij}$ および測定距離 $d_{ij}$ は誤差を含むので,
それらを平均した $\hat{v_i}$ は

$$
\frac{\hat{v_i}}{v_k} = \frac{1}{N} \sum_{i\neq j \neq k} \frac{x_{ij} d_{kj}}{x_{kj}d_{ij}} \\
$$

と定義できる.$d_{ii}$ のときは距離が0となりゼロ除算が発生するので,
$i\neq j \neq k$ としている.

一番出力の低い端末 $k$ の出力係数 $v_k$ を基準とすることで,
すべての端末において定格出力を守ることができる.

\section{隣接ノードでない端末間の同期,音圧校正,クロック校正}


互いに互いの信号を検出できなかったノード間での校正を考える.
図\ref{fig:networktopology}に隣接ノードではない端末を含むネットワークを示す.

\begin{figure}[tb]\centering
  \hspace{-2mm}\includegraphics[clip,width=1.1\hsize]{img/network_topology.png}
  \caption{networktopology}\label{fig:networktopology}
\end{figure}

このとき端末 A と端末 C 間では同期・測距ができていないが,
互いに端末 B とは同期・測距できているという状況である.

\subsection{同期}

時刻の基準となる
端末 $A$ がパルスを発した時間を $t_{AA}$ として
そのパルスが端末 $B$ に届いた時間は $t_{AB}$ とする.
また,
端末 $B$ がパルスを発した時間を $t_{AB}$ として
そのパルスが端末 $C$ に届いた時間は $t_{BC}$ とする.
そしてそれぞれの伝達時間を $d$ とすると図\ref{fig:reldelay}のようになる.

\begin{figure}[tb]\centering
  \hspace{-2mm}\includegraphics[clip,width=1.1\hsize]{img/rel_delay.png}
  \caption{reldelay}\label{fig:reldelay}
\end{figure}

このとき,まず端末 $C$ は端末 $B$ と同期して,
その後端末 $B$ と端末 $A$ の時刻ずれ情報をもとにさらに端末 $A$ との同期ができる.


\subsection{音圧校正}

端末 $A$ を基準に音圧校正を考えると

$$
\frac{v_B}{v_A} = \frac{x_{iB} d_{iB}}{x_{AB}d_{AB}} \\
\frac{v_C}{v_B} = \frac{x_{iC} d_{iC}}{x_{BC}d_{BC}} \\
$$

であるので

$$
\frac{v_C}{v_A} =
\frac{v_B v_C}{v_A v_B} =
\frac{x_{iB} x_{iC} d_{iB} d_{iC}}{x_{AB} x_{BC} d_{AB} d_{BC}}
$$

とすれば端末 A と端末 C の出力比率を求められる.

\subsection{クロック校正}


\section{同期・測距・校正手法のための制御システム}
同期のためには複数の端末がパルスを出し合わなければならないが,
いつどの端末がパルスを出すのか\ref{fig:TDMA},といったスケジューリングをどうするかについて述べる.

\begin{figure}[p]\centering
  \hspace{-2mm}\includegraphics[clip,width=1.1\hsize]{img/TDMA.png}
  \caption{TDMA}\label{fig:TDMA}
\end{figure}

図\ref{fig:network2}にシステム全体のネットワーク構成を示す.

\begin{figure}[p]\centering
  \hspace{-2mm}\includegraphics[clip,width=1.1\hsize]{img/network2.png}
  \caption{ネットワーク構成}\label{fig:network2}
\end{figure}

基本的には端末間の通信を中継するリレーサーバを中心としたスター型ネットワークである.
また,スピーカアレイに参加しない特別なノードとして,
計算用ノードと仮想音源を設定する制御用ノードがある.

次の図\ref{fig:flowchart}に同期・測距するまでのシステムフローを示す.

\begin{figure}[p]\centering
  \hspace{-2mm}\includegraphics[clip,width=1.1\hsize]{img/flowchart.png}
  \caption{同期制御フロー}\label{fig:flowchart}
\end{figure}

今回の実装では中継サーバが同期アルゴリズムを制御している.
すべてのコマンドはリクエスト-レスポンスで成り立っており,リクエストを受けた端末は必ずレスポンスを返さねばならない.
まず,中継サーバはスピーカアレイを構成する端末に対して ping コマンドを送信し,
アレイに参加できる端末を確認する.
次に,全端末に対して録音をするように beginRec コマンドを送信する.
そして,各端末の放つパルスが排他的になるように,
パルスを放つ端末ごとにstartPulse,beepPulse,stopPulseコマンドを繰り返し送信する.
startPulseとstopPulseコマンドは,
この時間区間内にいずれかの端末からパルスが発信されることを示すもので,
後にパルス位置を検出するときの計算量を減らすためのコマンドである.
beepPulseは任意の一台の端末に対して,パルスを送信するように促すコマンドである.

そしてすべての端末が互いに排他的にパルス発生し終えると,
最後にstopRecという録音終了コマンドを送信する.
その後,collectコマンドで各端末が録音したデータを集計し,
計算用サーバへ送信する.
計算用サーバは,それぞれの端末間のパルスの受信時刻を
先述の手法で検出し,相対信号伝達時間と相対距離計測,空間配置推定する.
その後,それらの情報を中継サーバを介して制御用端末へ送信する.

\section{仮想音源配置によるDBAPアレイスピーカ制御システムUI}

仮想音源を配置し制御するための端末のユーザインターフェースを示す(図\ref{fig:relpos}).

\begin{figure}[p]\centering
  \hspace{-2mm}\includegraphics[clip,width=1.1\hsize]{img/relpos.png}
  \caption{仮想音源配置UI}\label{fig:relpos}
\end{figure}

図のように推定した端末の分布図と,仮想音源を表示する.
仮想音源VSをドラッグすることで,DBAP法によって出力する振幅を計算し,各端末へ振幅を配信することで音像定位する.
当然,音を鳴らしながら音源を移動させることも可能である.


\section{P2Pリングネットワークを使った計算量の削減}
以前は
しかしながらパルス同定アルゴリズムのデバッグのしやすさの観点から、各端末に計算を分散して並列化による計算時間を削減するよりも、一台のパソコンで処理したほうが結果が比べやすいという開発上の観点から計算サーバーを導入しています
また、以前計算が停止した原因は私の書いたプログラムにゼロ除算を含む計算が生じる可能性があたったためで、nexus7でも計算自体は可能です。


\chapter{提案手法の検証実験}
\section{信号検出性能}
距離、ダイレクトパスなしなどの状況を変えて

\section{同期手法の有効性}

\section{同期手法の有効性}
\section{DBAP法の有効性}
\section{音圧校正手法の有効性}
\section{クロックずれ校正手法の有効性}
\section{隣接ノードでない端末間の各種手法の有効性}

\chapter{考察}
%研究全体の考察

\chapter{結論}
\section{本研究の成果}
%と,限定された点を明らかにしたり,
%さらに改善されるべき点を述べる
\section{本研究の展望}
%どうしたらもっとよくなるか
%どうしたら残ってる問題を解決できそうか.
\section{本研究の総括と結論}

\begin{thebibliography}{99}
%\begin{bibliography}{}
\addcontentsline{toc}{chapter}{参考文献}

% intro
\bibitem{11ubi}       Yu Zheng, Yanchi Liu, Jing Yuan, and Xing Xie, Urban Computing with Taxicabs, In Proc. of Ubicomp 2011, pp. 89--98, 2011.
\bibitem{surechigai}  末廣 創,佐藤 文明,すれ違い通信による情報伝搬モデルの特性評価,情報処理学会全国大会講演論文集 第72回(ネットワーク), 155-156, 2010-03-08, 2010.
\bibitem{clicker}     April R. Trees, and Michele H. Jackson. The learning environment in clicker classrooms: student processes of learning and involvement in large university‐level courses using student response systems. Learning, Media and Technology, vol.32, no.1, pp.21--40, 2007.
\bibitem{imakiku}     株式会社天問堂, Imakiku, \url{http://tenmondo.com/products/imakiku/index.html} (2015時点)
\bibitem{twitter}     後藤 真孝, WISS 2010 と WISS 2011 での改革, コンピュータソフトウェア (日本ソフトウェア科学会誌), Vol.29, No.4, pp.3--8, 2012.

% related
\bibitem{kawagoe14} 河越嵩介,神場知成,田中二郎.位置情報を利用した携帯端末への音声情報配信,情報処理学会第76回全国大会,4ZA-2,2014.
\bibitem{wfs}       Berkhout, Augustinus J., Diemer de Vries, and Peter Vogel. "Acoustic control by wave field synthesis." The Journal of the Acoustical Society of America 93.5 (1993): 2764-2778.
\bibitem{hoa}       Daniel, Jérôme. "Spatial sound encoding including near field effect: Introducing distance coding filters and a viable, new ambisonic format." Audio Engineering Society Conference: 23rd International Conference: Signal Processing in Audio Recording and Reproduction. Audio Engineering Society, 2003.
\bibitem{sfc}       Ise, Shiro. "A principle of sound field control based on the Kirchhoff-Helmholtz integral equation and the theory of inverse systems." Acta Acustica united with Acustica 85.1 (1999): 78-87.
\bibitem{paramsp}   青木茂明,清水一博,伊藤昂輝.パラメトリックスピーカを用いた再生時の音像定位.信学技報 EA研究会,vol.114,no.423, pp.33--38, 2015.
\bibitem{shibata13} 柴田一暁, 小野順貴, 亀岡弘和. 音の発信を利用したスマートフォンアレイの機器位置推定. 音講論 (秋), pp.591--592, 2013.
\bibitem{shibata14} 柴田一暁, 小野順貴, 亀岡弘和. 音の発信を利用したキャリブレーションに基づくアドホックマイクロホンアレイによる音源定位. 音講論 (春), pp. 707--710,2014.
\bibitem{Aoshima}   N. Aoshima, Computer-generated pulse signal applied for sound measurement, J. Acoust. Soc. Am., vol.69, no.5,1484--1488, 1981.


% system
\bibitem{dbap}      Trond Lossius, Pascal Baltazar, and Theo de la Hogue. DBAP–distance-based amplitude panning. Proc of ICMC 2009, pp.489--492, 2009.
\bibitem{tpsn}      Ganeriwal, Saurabh, Ram Kumar, and Mani B. Srivastava. "Timing-sync protocol for sensor networks." Proceedings of the 1st international conference on Embedded networked sensor systems. ACM, 2003.
\bibitem{Haas}      (After Helmut Haas's doctorate dissertation presented to the University of Gottingen, Gottingen, Germany as "Über den Einfluss eines Einfachechos auf die Hörsamkeit von Sprache;" translated into English by Dr. Ing. K.P.R. Ehrenberg, Building Research Station, Watford, Herts., England Library Communication no. 363, December, 1949; reproduced in the United States as "The Influence of a Single Echo on the Audibility of Speech," J. Audio Eng. Soc., Vol. 20 (Mar. 1972), pp. 145-159.)
\bibitem{pof}       Horner, Joseph L., and Peter D. Gianino. "Phase-only matched filtering." Applied optics 23.6 (1984): 812-816.
\bibitem{overwrap}  Rabiner, Lawrence R., and Bernard Gold. "Theory and application of digital signal processing." Englewood Cliffs, NJ, Prentice-Hall, Inc., 1975. 777 p. 1 (1975).

\bibitem{umezu11}   梅津直貴, 井ノ上寛人, 堀内恒, 佐藤美恵, 小黒久史, 春日正男.空間把握性に注目した音響案内システムの開発に関する研究.映像情報メディア学会技術報告 vol.35, no.39, pp.41--44,2011.
\bibitem{onzou}     平原達也, 蘆原郁, 小澤賢司, 宮坂榮一, 音と人間, 日本音響学会編, コロナ社, 2013.


\bibitem{raylei1877}
Lord Rayleigh, Acoustical Observations I,
Philosophical Magazine Series 5, vol. 3, Issue 20, pp.456--464, 1877.
%http://www.tandfonline.com/doi/abs/10.1080/14786447708639268?journalCode=tphm16
\bibitem{raylei1907}
Lord Rayleigh, On our perception of sound direction,
Philosophical Magazine Series 6, vol. 13, Issue 74, pp.214--232, 1907.
%http://www.tandfonline.com/doi/abs/10.1080/14786440709463595#abstract
%[Lord Rayleigh: On our perception of sound direction, Phil.Mag, 13,6th series,pp.214-232(1907)]
\bibitem{komatsu2014}
小松潤也, 塩田茂雄, センサ協調位置推定: 相互多辺測量法と多次元尺度構成法の精度比較, 信学技報, ASN 114(65), pp.127--132, 2014.
\bibitem{senkouon}
H. Haas, The Influence of a Single Echo on the Audibility of Speech,
Journal of Auditory Engineering Society, vol. 20, Issue 2, pp.146--159, 1972.
%http://www.aes.org/e-lib/browse.cfm?elib=2093
%播摩敏雄, 安倍幸治, 高根昭一, 曽根敏夫, 音像定位における先行音効果とエコー知覚の限界に関する考察, HIP 104(526), pp.13--18, 2004.
%http://ci.nii.ac.jp/els/110003272588.pdf?id=ART0003775596&type=pdf&lang=jp&host=cinii&order_no=&ppv_type=0&lang_sw=&no=1451791039&cp=
\bibitem{vbap}
Vikke Pulkki, Virtual Sound Source Positioning Using Vector Base Amplitude Panning, Journal of Audio Engineering Society, Vol. 45, Issue 6, pp.456--466, 1997.

%Trond Lossius, et al., DBAP - Distance based amplitude panning, International ComputerMusic Conference (ICMC). Montreal, 2009.
\bibitem{self_ac}
伊納洋祐, 吉田侑矢, 米澤朋子,	複数端末の音響的位置推定と同期による空間音響環境構築システムの提案,	ASJ 2014 autumn,	pp.1439--1440,	2014.
\bibitem{seigoufilter}
滑川俊彦,奥井重彦,衣斐信介,通信方式(第二版),森北出版,2012.
\bibitem{pulsecompress}
近藤倫正,, 実森彰郎,大橋由昌,計測・センサにおけるディジタル信号処理,昭晃堂,1993.
\bibitem{chordalg}
I. Stoica, R. Morris, D. Karger, M. F. Kaashoek, H. Balakrishnan.
Chord: A scalable peer-to-peer lookup service for internet applications.
ACM SIGCOMM Computer Communication Revies, vol. 31, issue 4, pp.149--160,
2001.
\bibitem{morimoto95}
Morimoto, Masayuki, and Kazuhiro Iida. A practical evaluation method of auditory source width in concert halls. Journal of the Acoustical Society of Japan (E) vol.16, no.2, pp.59--69, 1995.
\bibitem{barron81}
M. Barron, and A. H. Marshall. Spatial impression due to early lateral reflections in concert halls: the derivation of a physical measure. Journal of Sound and Vibration, vol.77, no.2, pp.211--232, 1981.
%音響工学基礎論
\bibitem{suehiro06}
末廣大地, 翁長博, 池田哲朗. 音楽ホールにおける音に包まれた感じに対応する物理指標の検討. 日本建築学会環境系論文集,no.599, pp.1--7, 2006.
\bibitem{cheun97}
Kyungwhoon Cheun. Performance of Direct-Sequence Spread-Spectrum RAKE Receivers with Random Spreading Sequences.
IEEE Transactions on communications, vol.45, no.9, pp.1130--1143, 1997.
\bibitem{honer84}
J. L. Horner, P. D. Gianino. Phase-only matched filtering. Applied optics, vol.23, no.6, pp.812--816, 1984.
%C言語によるディジタル無線通信技術
%アコースティック・イメージング
%原理が分かる・現場で使える信号処理

%http://sonove.angry.jp/about_localization.html
%http://acousticslab.org/psychoacoustics/PMFiles/Module07b.htm

\end{thebibliography}
%\end{bibliography}{}

\chapter*{謝辞}
\addcontentsline{toc}{chapter}{謝辞}

本研究を進めるにあたって,ご指導頂いた本学総合情報学部米澤朋子准教授,
副査を担当していただいた喜多千草教授,及び加藤隆教授に深く感謝いたします.
本学総合情報学研究科知識情報学専攻インタラクションの認知・メディア・文化プロジェクトの皆さんに,
様々な出来事の中で多くのことを学ばせていただいたことを感謝します.
堀井康史教授からは,電波工学の観点から測距に関するアドバイスを頂き,大きな指針となりました.
広兼道幸教授からは,結合振動子系の観点から同期に関するアイディアを頂き,本研究を支える動機付けとなりました.
伊藤俊秀教授からは,衛星工学の観点からパルス圧縮に関するアイディアを頂き,本研究の突破口となりました.
田中成典教授からは,測地測量の観点からアイディアを頂き,本研究の心の支えとなりました.
村田忠彦教授からは,マルチエージェントの観点からセル・オートマトンのアイディアを頂き,本研究の糧となりました.
小林孝史教授からは,データベースの観点からP2Pネットワークのアイディアを頂き,本研究の発展につながりました
吉田侑矢さん,石川佑樹さんには様々なサポートをいただきました.
ありがとうございました.
最後に,学部・大学院の 6 年間を通して,関西大学総合情報学部の多くの先輩・友人・後輩から様々な助けをいただけたことに感謝いたします.

\appendix

\def\thesection{付録\Alph{section}\\}


\chapter{納豆の菌糸の動画キャプチャ}


\begin{figure}[tb]\centering
\epsfxsize=10cm\epsffile{eps/natto.eps}
\caption{菌糸,33日目}\label{fig:kinshi33}
\end{figure}

\chapter{納豆を嫌がる人口比率と都道府県}

\begin{table}[tb]\centering
%\epxfxsie=10cm\epsffile{}
\caption{都道府県と納豆}\label{tab:todoufuken}
\begin{tabular}{lcc}
\hline
都道府県&すき&きらい \\
\hline
神奈川県&80&20 \\
大阪府&20&80 \\
\hline
\end{tabular}
\end{table}





\end{document}
