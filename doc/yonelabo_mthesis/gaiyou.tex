%
% $Id: blank.tex,v 2.0 2010-01-05 18:50:50+09 kobayasi Exp $
%
% Mar 21, 2001:  Revision Control Started!!
%
\documentclass[11pt]{jarticle}
\usepackage{newcent}             % PDFへの変換後の品質を高める
%
%\usepackage[doctor]{gaiyo}      % 博士論文要旨の場合
\usepackage[master]{gaiyo}      % 修士論文要旨の場合
%\usepackage{gaiyo}               % 卒業研究概要の場合
%\usepackage[junior]{gaiyo}      % 専門演習レポートの場合

%\usepackage{graphicx}
\usepackage[dvipdfmx]{graphicx}
\usepackage{epsf}
\usepackage{comment}

\title{複数の携帯端末による教室空間の空間音響環境構築手法の検討}
\id{14M7102}
\author{伊納洋佑}
\teacher{米澤朋子}

\begin{document}
\maketitle

\section{はじめに}

近年,スマートフォンやタブレット,ノートパソコンなどの高度な計算能力とネットワーク接続が可能なスマートデバイスが普及し,インターネットへの常時接続が可能になった.
こうした流れを受け,これらのスマートデバイスを互いに協調的に制御する研究が盛んになってきている\cite{shibata14}.
しかしながら,これらはセンシング技術が主体であり,環境に存在する人間に働きかけるものは少ない.
そこで本研究は,室内空間において,各々の所有するスマートデバイスを用いて平面配置のスピーカアレイを構築し,音像定位を行なうシステムを開発した.
このシステムは
例えば,授業中の生徒のスマートデバイスを用いることで,教室の特定の人間グループに対して注意喚起を促すことができる.
他にも,音像位置を動かすような効果を与えることができるので,音を用いた様々な活動に使えると考えられる.

\section{提案システム}

\subsection{DBAP法を用いた音像定位}

スマートデバイスのスピーカを用いた音像定位手法として,
仮想音源と各スピーカとの距離から距離減衰を算出し振幅による音像定位をするDBAP法を用いた\cite{dbap}.
これを用いるためには,デバイス間で音が遅延なく同期的に鳴らせる必要がある.
また,仮想音源と各端末間の距離が既知である必要がある.
さらに,各スピーカ間での音圧レベルが正しく制御できる必要がある.
以後にその手法を述べる.

\subsection{時刻同期と測距}

今回のようなスマートデバイス,あるいはセンサネットワークなどでの時刻同期の手法として,
音声パルスの往復による時刻同期手法が提案されている\cite{tpsn}.
これは二つの端末間で音声パルスを交互に鳴らし,
互いの端末で記録されたパルスの到来時刻に含まれる音速による伝搬時間の差を利用したものである.
標準大気圧下での音速を仮定することで,伝搬時間の差から相対距離を求めることができる.
この手法を用いれば,各端末が最低一回音声パルスを鳴らし,
互いのパルスの到来時刻とその差を計測することで時刻同期と相対距離計測が可能である.



\subsection{測距からの相対位置推定}

各端末の相対距離が判明していれば,多次元尺度構成法などを用いて空間上での相対的な分布を推定することができる.
今回の場合は,各端末間の距離は測定誤差を含み,他にも距離が離れすぎていて音声パルスが検出できない可能性があるので,
そのような時にも使用できる非計量多次元尺度構成法を用いた.
これは計測相対距離と推定位置の相対距離の差を取る誤差関数を,最少化する非線形最適化問題として定式化することで,最急降下法などで解く手法である.



\subsection{信号検出}

距離を精度よく計測するためには,パルスの到来時刻を正しく検出する必要がある.
しかしながら室内環境では環境雑音も多く,他にも壁や天井などに反射によるマルチパスの問題が生じる.
そこで音声パルスには,環境雑音にも強い擬似雑音系列を用いた直接スペクトル拡散方式によるパルス圧縮を用いた.
そしてマルチパス環境下でのパルス位置推定手法として,Rake受信器にも使われる伝送路測定用信号と計測信号の二つを用意する手法を応用し,
それらの新郷を互いに短時間相互相関にかけることで,厳密なパルス到来時刻の同定を試みた.



\subsection{校正}

今回の場合,異種端末間でのハードウェア・ソフトウェアに起因する差異の影響を考慮する必要がある.
例えば,スピーカ出力の異なる端末間ではDBAP法は正しく使用できない.
他にも,システムクロックが異なる場合,同期後時間が経つと次第にズレが大きくなってきてしまうなどの問題がある.
本稿ではそのような端末間の差異の校正手法についても検討する.


\subsection{同期・測距・校正手法のための制御システム}

以上の技術を用いればスピーカアレイを構築できるが,これを具体的に同制御するのかという問題がある.
今回は中央サーバによる通信の中継と計算サーバを導入したスター型ネットワーク構成と,
中央サーバを必要としないChrodアルゴリズム\cite{chordalg}で構築したP2Pリングネットワークによる構成の二つを提案し,
それぞれの利点と問題点を述べる.


\section{おわりに}
本稿では,多数の人間が存在する閉鎖空間内のパラレルコミュニケーション手法として,音源定位した複数の音を用いることを考え,
複数参加者各自が所有するスマートデバイスを用いてスピーカアレイを構築し,音源配置した音声情報を提供するシステムを提案した.
これにより,音源定位をしたい空間内座標に近い端末を複数選択し,それぞれの端末の振幅パニングで,所望の位置に近い音像定位により音声情報提供することができる.
また,この手法を用いて同期したスマートデバイスによるマルチチャネルパニングにより発生させた音は,3つの発音端末の三角形の外の受聴者には三角形内に定位されることを確認した.
今後は,各端末の特性としてマイクやスピーカ位置,音響特性を考慮することや,端末数をどの程度まで増やすべきかの議論を行っていきたい.


\begin{thebibliography}{10}

\bibitem{shibata14} 柴田一暁, 小野順貴, 亀岡弘和. 音の発信を利用したキャリブレーションに基づくアドホックマイクロホンアレイによる音源定位. 音講論 (春), pp. 707--710,2014.
\bibitem{dbap}      Trond Lossius, Pascal Baltazar, and Theo de la Hogue. DBAP–distance-based amplitude panning. Proc of ICMC 2009, pp.489--492, 2009.
\bibitem{Haas}      (After Helmut Haas's doctorate dissertation presented to the University of Gottingen, Gottingen, Germany as "Über den Einfluss eines Einfachechos auf die Hörsamkeit von Sprache;" translated into English by Dr. Ing. K.P.R. Ehrenberg, Building Research Station, Watford, Herts., England Library Communication no. 363, December, 1949; reproduced in the United States as "The Influence of a Single Echo on the Audibility of Speech," J. Audio Eng. Soc., Vol. 20 (Mar. 1972), pp. 145-159.)
\bibitem{tpsn}      Ganeriwal, Saurabh, Ram Kumar, and Mani B. Srivastava. "Timing-sync protocol for sensor networks." Proceedings of the 1st international conference on Embedded networked sensor systems. ACM, 2003.
\bibitem{chordalg} I. Stoica, R. Morris, D. Karger, M. F. Kaashoek, H. Balakrishnan.Chord: A scalable peer-to-peer lookup service for internet applications.ACM SIGCOMM Computer Communication Revies, vol. 31, issue 4, pp.149--160,2001.

\end{thebibliography}

\end{document}
