近年,スマートフォンやタブレット,ノートパソコンなどの高度な計算能力とネットワーク接続が可能なスマートデバイスが普及し,インターネットへの常時接続が可能になった.
こうした流れを受け,これらのスマートデバイスを互いに協調的に制御する研究が盛んになってきている\cite{shibata14}.
しかしながら,これらはセンシング技術が主体であり,環境に存在する人間に働きかけるものは少ない.
そこで本研究は,室内空間において,各々の所有するスマートデバイスを用いて平面配置のスピーカアレイを構築し,音像定位を行うシステムを開発した.
このシステムは
例えば,授業中の生徒のスマートデバイスを用いることで,教室の特定の人間グループに対して注意喚起を促すことができる.
他にも,音像位置を動かすような効果を与えることができるので,音を用いた様々な活動に使えると考えられる.
