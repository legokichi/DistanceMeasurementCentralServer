\chapter*{発表文献リスト}
\addcontentsline{toc}{chapter}{発表文献リスト}

\begin{itemize}

  \item[] ジャーナル論文
  \begin{enumerate}
    \item 中祐介, 伊納洋佑, 吉田直人, 米澤朋子, 身体動作・環境音のオノマトペを含むテキストコミュニケーション手法の検討, HI学会論文誌, vol.17, no.2, pp 97--106, 2015.
  \end{enumerate}

\begin{comment}
  \item[] 国際会議論文
  \begin{enumerate}
    \item hoge
  \end{enumerate}
\end{comment}


  \item[] 国内会議論文
  \begin{enumerate}
    \item 伊納洋佑, et al. 複数の携帯端末による教室空間の空間音響環境構築手法の検討. 研究報告コンピュータビジョンとイメージメディア (CVIM), 2014, 2014.7: 1-4.
    \item 伊納洋佑, et al. 複数の携帯端末による教室空間の空間音響環境構築手法の検討 (人体・動作の認識と理解, 福祉と共生, 国際会議報告). 電子情報通信学会技術研究報告. MVE, マルチメディア・仮想環境基礎, 2014, 113.403: 41-44.
    \item 岡本直也; 伊納洋祐; 米澤朋子. ディジタル画像への温感付与による非実体の体感システム提案 (第 109 回ヒューマンインタフェース学会研究会 高齢者, 障がい者支援技術および一般). ヒューマンインタフェース学会研究報告集, 2014, 16: 33-36.
    \item 河口拓貴, et al. 上体の重心移動を伴う身体動作による音楽演奏時のリズム生成手法の提案 (人体・動作の認識と理解, 福祉と共生, 国際会議報告). 電子情報通信学会技術研究報告. PRMU, パターン認識・メディア理解, 2014, 113.402: 49-52.
    \item 石野力, et al. 複数パラメトリックスピーカを用いた一対多コミュニケーション手法の提案 (人体・動作の認識と理解, 福祉と共生, 国際会議報告). 電子情報通信学会技術研究報告. MVE, マルチメディア・仮想環境基礎, 2014, 113.403: 53-58.
    \item 石野力; 伊納洋祐; 米澤朋子. 空間指向性を含む繰り返し音楽の制御と演奏効果の検証. 情報処理学会研究報告.[音楽情報科学], 2015, 2015.18: 1-6.
    \item 中祐介, et al. 身体動作・環境音のオノマトペを含むテキストコミュニケーション手法の検討 (特集論文 「いい加減」なインタフェース) ヒューマンインタフェース学会論文誌 The transactions of Human Interface Society 17(1-4), 97-106, 2015
  \end{enumerate}

  \item[] その他
  \begin{enumerate}
    \item 国際バーチャルリアリティ学生コンテスト 「茶禅空」審査員特別賞受賞,2013.
  \end{enumerate}

\end{itemize}
